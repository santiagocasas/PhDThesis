% Chapter 1

\chapter{Overview of Standard Cosmology} % Main chapter title

\label{Overview} % For referencing the chapter elsewhere, use \ref{Chapter1} 

%----------------------------------------------------------------------------------------

% Define some commands to keep the formatting separated from the content 
\newcommand{\keyword}[1]{\textbf{#1}}
\newcommand{\tabhead}[1]{\textbf{#1}}
\newcommand{\code}[1]{\texttt{#1}}
\newcommand{\file}[1]{\texttt{\bfseries#1}}
\newcommand{\option}[1]{\texttt{\itshape#1}}

%----------------------------------------------------------------------------------------

\section{The framework of General Relativity}

\subsection{The equivalence principle and geometry}
\begin{itemize}
\item The equivalence principle
\item Manifolds and local flatness
\item Physical relation equivalence principle -> local flatness
\item The metric formalism
\item The Levi-Civita connection
\item Dipheomorphism invariance
\end{itemize}


\subsection{The Einstein-Hilbert action and the field equations}

\begin{equation}
S = \int \textrm{d}x^4 \sqrt{-g}( R - \Lambda)
\end{equation}

\begin{equation}
R_{\mu \nu} + \frac{1}{2}g_{\mu \nu} R + g_{\mu \nu} \Lambda = 8 \pi G T_{\mu \nu}
\end{equation}

\begin{itemize}
\item Bianchi identities and the conservation of $T_{\mu \nu}$
\item Lovelock's theorem
\end{itemize}

\subsection{The Newtonian limit}

\begin{itemize}
\item Gravitational potentials $\Phi$ and $\Psi$.
\item Very simple explanation of Gauge choice
\item Newtonian Gauge
\item Smallness of gravitational potentials
\item Remark for later that in GR $\Psi=\Phi$
\item Weyl potential (lensing potential, null geodesics)

\end{itemize}

%----------------------------------------------------------------------------------------

\section{The standard cosmological model}

\begin{itemize}
\item small discussion on how to define a cosmology
\item Observer along timelike geodesic and foliation of spacetime
\end{itemize}

\subsection{Friedmann equations}
 \begin{itemize}
 \item Friedmann-Lemaître-Robertson-Walker metric
 \item 00 and ii components of Einstein equation
 \item First and second Friedmann equations
 \item Several ways of writing Hubble function for different species
 \item Background expansion, , RDE, MDE, w(z)
 \end{itemize}

\subsection{Distances}

\begin{itemize}
\item definition of distances
\item definition of Hubble time 
\item Horizons?
\end{itemize}


\subsection{The $\Lambda$CDM paradigm}

\begin{itemize}
\item Reasons for CDM
\item Reasons for Lambda
\item Baryons, neutrinos and other species
\end{itemize}

\subsection{The cosmological constant problem}

\begin{itemize}
\item small discussion on old and new cosmological constant problem
\item fine tuning, naturalness, basic definitions
\end{itemize}

\subsection{Linear perturbations}

\begin{itemize}
\item Differentiate between linear perturbations in different eras, just shortly
\item Linear perturbations in matter dominated eras, newtonian gauge
\item Fluid equations, full and linearized
\end{itemize}


%----------------------------------------------------------------------------------------

\section{Early Universe}

\subsection{Cosmic Microwave Background Radiation}
\begin{itemize}
\item Short introduction and importance of CMB
\item Important constraints on parameters coming from CMB
\item Constrain CDM alone
\item Constrain initial power amplitude and tilt
\item Constrain relativistic degrees of freedom
\end{itemize}

\subsection{Inflation}
\begin{itemize}
\item Inflation as a paradigm
\item Flatness and horizon problems
\item Inflation produces almost scale invariant spectrum
\end{itemize}





