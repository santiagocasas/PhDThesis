% Chapter 1

\chapter{Overview of Standard Cosmology} % Main chapter title

\label{Overview} % For referencing the chapter elsewhere, use \ref{Chapter1} 

%----------------------------------------------------------------------------------------

% Define some commands to keep the formatting separated from the content 
\newcommand{\keyword}[1]{\textbf{#1}}
\newcommand{\tabhead}[1]{\textbf{#1}}
\newcommand{\code}[1]{\texttt{#1}}
\newcommand{\file}[1]{\texttt{\bfseries#1}}
\newcommand{\option}[1]{\texttt{\itshape#1}}

%----------------------------------------------------------------------------------------

\section{GR}


%----------------------------------------------------------------------------------------

\section{Early Universe}



\subsection{Inflation}



\subsection{CMB}



\subsection{Dark Ages}


\subsection{BBN}
 

%----------------------------------------------------------------------------------------

\section{Distances}



\subsection{Horizons}



%----------------------------------------------------------------------------------------

\section{Perturbations}

\subsection{LCDM}


\begin{itemize}
\item Chapter 1: Introduction to the thesis topic
\item Chapter 2: Background information and theory
\item Chapter 3: (Laboratory) experimental setup
\item Chapter 4: Details of experiment 1
\item Chapter 5: Details of experiment 2
\item Chapter 6: Discussion of the experimental results
\item Chapter 7: Conclusion and future directions
\end{itemize}
This chapter layout is specialised for the experimental sciences.



\subsection{Cosmological constant}



%----------------------------------------------------------------------------------------



%----------------------------------------------------------------------------------------

\section{Dark Energy}


%----------------------------------------------------------------------------------------



\subsection{Quintessence}




\subsection{Neutrinos}


\subsection{Dark Matter}

