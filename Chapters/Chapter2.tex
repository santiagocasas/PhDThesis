\chapter{Dark Energy and Modified Gravity} % Main chapter title

\label{DE-MG} % For referencing the chapter elsewhere, use \ref{Chapter1} 

%----------------------------------------------------------------------------------------


%----------------------------------------------------------------------------------------


\section{Universal coupling to matter}

\subsection{Quintessence}

\subsection{f(R) Theories}

\subsection{Horndeski Theory}

\begin{itemize}
\item Most general single scalar field theory, second order equations of
motion, 4 dimensions, ghost-free
\item 5 Lagrangians which can be parametrized with the $\alpha$ functions
that give more physical insight.
\item Quasistatic limit yields modifications of gravity that depend only
on time and $k^{2}$.
\end{itemize}

In a homogeneous and isotropic universe, the distance element
assumes the form: 
\begin{equation}
ds^{2}=a(\tau)[-(1+2\Psi)d\tau^{2}+(1-2\Phi)\delta_{ij}dx^{i}dx^{j}]\,,
\end{equation}
where the conformal time $\tau$ is related to the cosmic time $t$
by $d\tau=d t/a(\tau)$ and $\Psi$ and $\Phi$ are the two scalar
gauge invariant gravitational potentials, related (modulo sign) to
the Bardeen potentials.

Horndeski models include (almost) all theories
with second order equations of motion in the fields and can be generically
described with five functions of time. They include a variety of particular
models studied in literature, characterised by a universal coupling
to matter fields.

The Lagrangian can be written as ...

where ... is the matter Lagrangian and ... are arbitrary functions
of the scalar field $\phi$, whose canonical kinetic term is $X=-\frac{1}{2}\phi^{;\mu}\phi_{;\mu}$;
the latter is also coupled to the Ricci scalar and the Einstein tensor.

In the following, we will adopt the conformal Newtonian gauge in the
notation of \citep{MaBertschinger}, so that the gauge invariant potentials
$\Psi$ and $\Phi$ are also equal to the gravitational potentials
in the Newtonian limit. In this formulation, the cold dark matter
(CDM, c) perturbation equations for the density contrast $\delta$
and the velocity divergence $\theta$ can be written in Fourier space
in the usual way: 
\begin{align}
\dot{\delta}(\tau,\vk k) & =-(\theta(\tau,\vk k)-3\Phi(\tau,\vk k))\,\,,\\
\dot{\theta}(\tau,\vk k) & =-H\theta(\tau,\vk k)+k^{2}\Psi(\tau,\vk k)\,\,.
\end{align}
Derivating the first of the above equations with respect to $\tau$
and eliminating $\dot{\theta}$ and $\theta$ using both equations,
one obtains the usual second order equation in time for the density
contrast: 
\begin{equation}
\ddot{\delta}(\tau,\vk k)=-(H\dot{\delta}(\tau,\vk k)-3H\dot{\Phi}(\tau,\vk k)+k\text{\texttwosuperior}\Psi(\tau,\vk k)-3\ddot{\Phi}(\tau,\vk k))\,\,.\label{eq:deltadotdot}
\end{equation}

%For dimensional reasons \comment{VP: what do you mean? dimensional
%reasons cannot tell you anything about the amplitude of any quantity}
%and since we expect that the potentials vary only strongly \comment{VP:
%what does it mean to vary only strongly? only or strongly?} at scales
%compared to the horizon $H^{-1}$, we must have that \comment{VP:
%do you mean you are assuming it?} $\ddot{\Phi}\sim H\dot{\Phi}\sim H^{2}\Phi$
%and besides we also expect $\Psi\sim\Phi$ at least at the linear
%level (\comment{VP: not sure how this can be true in MG theories
%nor why we need it at all. I would delete. In general only use precise
%and motivated statements or don't say it at all.}). If we then divide
%Eq. \ref{eq:deltadotdot} by $H^{2}$, the potentials $\Phi$ can
%be neglected with respect to the term containing $k^{2}/H^{2}$, since
%for subhorizon scales ($k\gtrsim0.01h/\mbox{Mpc})$ it is many orders
%of magnitude bigger. Therefore we obtain: \comment{VP: I would completely
%delete all this text after eq.(4) and just say: in the Newtonian limit
%(or directly quasi-static), for which $k/H\ll1$, Eq. 4 reduces to:}
In the following, we will stay within the quasi-static
limit, i.e restrict to scales much smaller than the cosmological horizon
($k/aH\gg1$) and well inside the Jeans length of the scalar field
$c_{s}k\gg1$, so that terms containing $k$ dominate over terms containing
time derivatives. In this case, Eq. \ref{eq:deltadotdot} reduces
to: 
\begin{equation}
\ddot{\delta}(\tau,\vk k)+H\dot{\delta}(\tau,\vk k)+k^{2}\Psi(\tau,\vk k)=0\,.\label{eq:growth-rate-Psi}
\end{equation}
The gravitational potential is then obtained from the Poisson equation,
that relates the gravitational potential to matter and relativistic
species in the universe, by perturbing the Einstein field equations
to first order.

In Horndeski theories, in the quasi-static limit, the deviation
of the gravitational potentials from General Relativity has a known
scale dependence in the Poisson equation, given by \citep{...}: 
\begin{equation}
\mu(k,a)\equiv-\frac{2k^{2}\Psi}{3\Omega_{m}\delta_{m}}=h_{1}\left(\frac{1+(k/k_{*})^{2}h_{5}}{1+(k/k_{*})^{2}h_{3}}\right)\label{eq:Yfunc}
\end{equation}
where $h_{1},h_{3},h_{5}$ are functions of time only. A similar expression,
with different time dependent coefficients $h_{2},h_{4}$, holds for
the gravitational slip $\eta$: 
\begin{equation}
\eta(k,a)\equiv-\frac{\Phi}{\Psi}=h_{2}\left(\frac{1+(k/k_{*})^{2}h_{4}}{1+(k/k_{*})^{2}h_{5}}\right)\,.\label{eq:etaFunc}
\end{equation}
In this work we will consider all $h_{i}$ functions as constants,
for simplicity, although in general they are all functions of time.
%since we are expecting them not to affect so much the evolution of growth with respect to%a fiducial LCDM model. %Once a specific model of the Horndeski class%is chosen, the $h_i$ functions and their time dependence can be calculated%and its approximated constant value during the time of structure growth%can be estimated. The
$k_{*}$ is an arbitrary pivot scale, chosen at large scales close
to the horizon, in order to fulfil the quasi-static limit.

As it can be seen from eqns. \ref{eq:Yfunc} and \ref{eq:etaFunc},
the functions $h_{5}$ and $h_{3}$ are degenerate as well as $h_{4}$
and $h_{5}$. %Since $h_{5}$ appears in the denominator as well as%in the numerator of both equations, a combination of both such as%$\mu(1+\eta)$ also retains the smae structure. %This is of importance%since such a combination would appear in the modified weak lensing%potential. The
$\Lambda\textrm{CDM}$ case is recovered when $\mu=\eta=1$, which
implies $h_{5}=h_{4}=h_{3}$ and an amplitude $h_{1}=h_{2}=1$.

In the following, we will focus on CDM perturbations and deal only
with the function $\mu(k,a)=\mu(k)$ (although perturbations require
two functions of time and scale to fully specify the model). In particular,
we will test the following cases: 
\begin{enumerate}
\item Scale independent $\mu$ with modified amplitude: $h_{5}=h_{3}$ and
$h_{1}>1$ or $h_{1}<1$. 
\item Scale dependent $\mu$ with unity amplitude: $h_{1}=1$ and $h_{5}>h_{3}$
or $h_{5}<h_{3}$. 
\item Scale dependent $\mu$ with modified amplitude: $h_{1}\neq1$ and
$h_{5}>h_{3}\mbox{ or }h_{5}<h_{3}$. 
\end{enumerate}


In logarithmic space $\mu(k)$ is very close to a step function, as
shown in figure \ref{fig:Y(k)-function}, so that we will only use
values for the $h_{i}$ in which the difference of the minimum and
the maximum is of the order of 15\%, since otherwise we would impose
a rather unrealistic structure formation.


\subsection{Effective Field Theory of Dark Energy}

\begin{itemize}
\item Another approach to construct a general model for dark energy based
on the allowed symmetries in the action.
\item Can be parametrized with the $\alpha$ functions.
\item Allows in a simple way to create theories beyond Horndeski and to
accomodate non-universal couplings.
\end{itemize}

\section{Non-universal coupling}

\begin{itemize}
\item Motivate dark energy, the coincidence problem and a dark sector interaction
\item Explain why baryons should be uncoupled (Solar system constraints)
\end{itemize}

\subsection{Coupled Dark Energy}

\begin{itemize}
\item Introduce coupling at the energy momentum tensor
\item Linear perturbations, gravitational bias DM-baryons, signatures in
LSS, non-linear perturbations
\item Talk about present constraints
\end{itemize}

\subsection{Growing Neutrino Quintessence}
\begin{itemize}
\item Motivate coupling to neutrinos (energy scale of DE and coincidence
problem), particle physics seesaw mechanism perspective, non-relativistic
neutrinos
\item Background equations, linear perturbations and instabilities, failure
of linear theory to take into account structures, lumps and backreaction
\item Non-linear perturbations, simulations, neutrino lump dynamics, heating
of neutrinos
\end{itemize}

\section{Non-local gravity}
\begin{itemize}
\item Non-local terms in the action, like inverses of the box operator acting
on the Ricci scalar, are motivated by quantum gravity corrections
\item Non-local expressions can be localized in terms of scalar fields.
\item There are the same number of parameters as in LCDM, instead of $\Lambda$
there is a mass parameter.
\item Seems to give a very good fit to present observations.
\end{itemize}