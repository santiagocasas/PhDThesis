\chapter{Observables and Experiments in Cosmology} % Main chapter title

\begin{itemize}
\item Can be split roughly into two big categories and each one into smaller
categories. (not taking into account more futuristic probes)
\item The two big categories are: Early Universe and Late Universe.
\item In the Early Universe there is the CMB temperature map, the CMB polarization
maps and Big Bang Nucleosynthesis constraints. Soon there will be
Gravitational Wave Background constraints.
\item In the late universe there are broadly two categories: Background
expansion and Structure formation. 
\item Background expansion includes: distance-redshift measurements, geometric
measurements
\item Structure formation includes: Galaxy Clustering, Weak Lensing, CMB
Lensing.
\end{itemize}


\label{ObsExp} % For referencing the chapter elsewhere, use \ref{ObsExp} 

%----------------------------------------------------------------------------------------


%----------------------------------------------------------------------------------------



\section{Clustering of galaxies}

\begin{itemize}
\item Important observable is the two point correlation function of galaxy
positions and its Fourier transform, the power spectrum.
\item Galaxies are supposed to trace the distribution of dark matter density.
The difference between galaxy density distribution and DM distribution
is called the bias.
\item Galaxy clustering measures effectively the Poisson equation and therefore
$\delta$ and modifications to it. Through BAO measurements and redshift
bins it also measures the growth rate $f$ and geometrical quantities
such as $H$ and $D_{A}$.
\item The linear power spectrum can be calculated exactly, while at small
scales (large $k$) the non-linearities coming from the Vlasov-Poisson
system become important.
\end{itemize}


\section{Weak gravitational lensing}

\begin{itemize}
\item The fundamental observable is the Weyl Power spectrum, which is the
power spectrum of the Weyl potential $\Phi+\Psi$.
\item The Weyl power spectrum can on one side be related to the matter density
fluctuations for a specific model that depends and on the other side
it is related to the shear and convergence maps of the distortion
of galaxy shapes.
\item Weak lensing is a powerful measurement of large scale structure, since
it is directly sensitive to dark matter and does not depend on details
of baryon formation. On the other hand, it does not give so much information
about geometrical quantities or the ratio between $\Omega_{b}$ and
$\Omega_{c}$.
\item It can be contaminated with lots of systematics, like intrinsic alignments
and instrument noise. For being useful, it needs a well determined
non-linear matter power spectrum.
\end{itemize}

\section{Future Experiments}

\subsection{Euclid}
\begin{itemize}
\item This space mission will measure \textasciitilde{}100 Million spectroscopic
redshifts of galaxies and \textasciitilde{}10\textasciicircum{}9 photometric
images of galaxies.
\item It will be a galaxy Clustering and Weak Lensing probe. Combining both
observations it will yield unprecedented constraints on cosmological
parameters.
\item For using all its data, we will need to be able to calculate the linear
and non-linear matter power spectrum in many different cosmological
models.
\item The fiducial bias and other systematics have to be removed by doing
N-body and End-to-End simulations.
\end{itemize}

\subsection{SKA 1 \& 2}
\begin{itemize}
\item SKA1 will be the first stage of an array of radio telescopes. Among
many interesting observations, it is capable of measuring redshifts
and shapes of galaxies.
\item SKA2 will be a major upgrade planned for \textgreater{}2025. 
\item It strength will be that it not only determines Galaxy Clustering
and Weak Lensing, but also its cross-correlation with more futuristic
probes, like Ly-alpha and Intensity mapping.
\end{itemize}

\subsection{DESI}
\begin{itemize}
\item It is a ground based observation and will be able to measure millions
of redshifts of galaxies. 
\item It can discriminate between bright galaxies, red galaxies and quasars.
\item It will only be useful for Galaxy Clustering.
\end{itemize}