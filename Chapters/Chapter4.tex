\chapter{Statistics in Cosmology} % Main chapter title

\label{Statis} % For referencing the chapter elsewhere, use \ref{Chapter1} 

%----------------------------------------------------------------------------------------

\section{Correlation functions and the power spectrum}
\begin{itemize}
\item Gaussian random variables
\item Statistical isotropy and homogeneity
\item Correlation functions and power spectrum
\end{itemize}

\section{Likelihood and the Bayesian approach}
\begin{itemize}
\item Bayes Theorem 
\item Likelihood
\item Priors, posteriors and evidence
\end{itemize}

\section{Fisher Matrix formalism}
\begin{itemize}
\item Approximation of the Gaussian likelihood at the minimum
\item Ellipses and confidence contours
\item Marginalization and maximation
\end{itemize}

\section{The Fisher Matrix Code}
\begin{itemize}
\item This serves as a manual for the Fisher Tools code.
\end{itemize}

\subsection{Methodology of Fisher forecasts}
\begin{itemize}
\item Derivatives of cosmological functions
\item Integration or summation in $k$ or $\ell$ space
\item Alcock-Paczynski effect
\item Redshift binning 
\item Shape, redshift and nuisance parameters
\item Marginalization
\end{itemize}



\subsubsection{The observed power spectrum}

The observed power spectrum is built from the theoretical power spectrum
$P(k,z)$ which depends on a vector of fundamental cosmological parameters
$\theta_{i}$ and the following contributions:
\begin{enumerate}
\item The contribution from redshift space distortions and bias $b(z)$,
known as the Kaiser formula, where $\beta(z)=f(z)/b(z)$ and $f(z)$
is the growth rate of density perturbations. Terms shown in green
below. 
\item The geometrical effect of the change of cosmological parameters (the
so called Alcock-Paczynski-effect), where $H(z)$ is the hubble function
and $D_{A}(z)$ is the angular diameter distance evaluated at a different
set of cosmological parameters as $H_{fid}(z)$ and $D_{A}(z)_{fid}$
which are defined at the fiducial parameter values (the ratio $H_{fid}(z)/H(z)$
is clearly 1 if both are evaluated at the fiducial values). If this
effect is taken fully into account, then $k$ and $\mu$ also have
to be affected by a geometrical term when evaluating the power spectrum
at a different set of parameters. (see the AP-effect section \ref{sub:The-Alcock-Paczynski-Effect}
below). These terms are shown in red below.
\item The damping due to redshift errors $\sigma_{z}$ and non-linear peculiar
velocity dispersions $\sigma_{v}(z)$, which is the first order correction
term to the Kaiser formula. Terms shown in magenta below. 
\item Extra shot noise due to observational effects, which can be included
in general with the term $P_{s}(z)$. Term in blue below.
\end{enumerate}
\begin{equation}
P_{{\rm obs}}\left(z,k,\mu;\theta\right)=\textcolor{blue}{P_{{\rm s}}(z)}+\textcolor{red}{\frac{D_{A}^{2}(z)_{ref}H(z)}{D_{A}^{2}(z)H(z)_{ref}}}{\color{green}b^{2}(z)\left(1+\beta(z){\color{red}\mu}^{{\color{red}2}}\right)^{2}}P({\color{red}k},z){\color{magenta}e^{-k^{2}\mu^{2}(\sigma_{z}^{2}/H(z)+\sigma_{v}^{2}(z))}}\label{eq:obs-power-basic}
\end{equation}


If one wishes to place constraints on the normalization of the power
spectrum, $\sigma_{8}$, then $P(k,z)$ can be written as:

\begin{equation}
P(k,z)=\sigma_{8}^{2}\tilde{P}(k,z)\label{eq:normalizedPk}
\end{equation}
where $\tilde{P}(k,z)$ is the normalized power spectrum. The normalization
$s_{R}$of the power spectrum (which is equivalent to the variance
smoothed over a scale $R$) is given by:
\begin{equation}
s_{R}=\frac{1}{2\pi^{2}}\int_{k_{min}}^{k_{max}}k^{2}P(k,z=0)W_{R}^{2}(kR)\label{eq:normalizationPk}
\end{equation}


The window function smoothes $P(k,z)$ over the scale $R$ and has
the form: $W(x)=3(\sin(x)-x\cos(x))/x^{3}$, where $x$ is a dimensionless
variable. This term comes from assuming spherical symmetry and performing
the angle integration of the 3-D power spectrum $P(\vec{k})$. If
one chooses $R=8\mbox{h}^{-1}\mbox{Mpc}$ and $[k]=\mbox{h}/\mbox{Mpc}$,
then we recover the usual $s_{R}=\sigma_{8}^{2}$. The normalization
$\sigma_{8}$ is a quantity defined only in linear theory, which means
that in eqn. \ref{eq:normalizationPk} either $P(k,z)$ is linear
or the integration limits are chosen in such a way that only the linear
normalization is taken into account, so that usually one should take
$k_{max}\approx0.1$. The normalized power spectrum is simply: 
\begin{equation}
\tilde{P}(k,z)=\frac{1}{s_{R}}P(k,z)
\end{equation}


Due to the of the fact that $\sigma_{8}$ is a linear quantity, we
can use it as an independent cosmological parameter, because we can
rescale it arbitrarily before it is compared with observations. It
is then completely degenerate with the initial primordial amplitude
of the power spectrum, $\mathcal{A}_{s}$, and one can replace one
parameter by the other, fixing just one of both.\footnote{One has to emphasize that this can only be done for linear power spectra,
because if one takes nonlinear density fluctuations into account for
calculating a non-linear $P(k,z)$, a different primordial amplitude
of fluctuations has a non-linear effect at small scales in the so
called ``non-linear bump''. The degeneracy between $\mathcal{A}_{s}$
and $\sigma_{8}$ is then broken.}

Furthermore if one wishes to express a linear power spectrum $P(k,z)$
in terms of the primordial power spectrum, the growth $G(z)$ and
the transfer functions $T(k)$, one can write eqn. \ref{eq:normalizedPk}
as: 
\begin{equation}
P(k,z)=\frac{\sigma_{8}^{2}}{s_{R}}G^{2}(z)T^{2}(k)k^{ns}\mathcal{A}_{s}
\end{equation}
being careful to take either $\sigma_{8}$ or $\mathcal{A}_{s}$ as
a primordial parameter and not both. In the case in which $\mathcal{A}_{s}$
is the cosmological parameter, the ratio $\frac{\sigma_{8}^{2}}{s_{R}}$
should be 1. The growth function $G(z)$ depends on the cosmological
parameters $\theta_{i}$ and is defined to be normalized to unity
today: $G(z=0)=1$. The quantity $\sigma_{8}^{2}G^{2}(z)\equiv\sigma_{8}(z)$
can be defined accordingly and can be used as an independent and time-dependent
cosmological parameter too. 


\subsubsection{Derivatives of the observed power spectrum}

For the calculation of the Fisher matrix, one needs to take derivatives
of the logarithm of the observed power spectrum $P_{obs}(k,z,\mu)$
at the central value $\bar{z}$ of each redshift bin. (see equation
for the Fisher matrix). There are two options here in terms of the
practical and numerical approach: 
\begin{itemize}
\item One calculates numerically: 
\begin{equation}
\left.\frac{\mbox{d}\ln P_{{\rm obs}}\left(\bar{z},k,\mu;\theta_{i}\right)}{\mbox{d}\theta_{i}}\right|_{fid}=\frac{P_{{\rm obs}}\left(\bar{z},k,\mu;\theta_{i}^{+}\right)-P_{{\rm obs}}\left(\bar{z},k,\mu;\theta_{i}^{-}\right)}{2\varepsilon\theta_{i}^{fid}\times P_{{\rm obs}}\left(\bar{z},k,\mu;\theta_{i}^{fid}\right)}
\end{equation}
where $\theta_{i}^{+}$, $\theta_{i}^{-}$ represent the parameter
$\theta_{i}$ evaluated at $\pm\varepsilon$ around the fiducial value
$\theta_{i}^{fid}$:
\begin{equation}
\theta_{i}^{\pm}=\theta_{i}^{fid}(1\pm\varepsilon)
\end{equation}
However this assumes that all functions inside $P_{{\rm obs}}(\bar{z},k,\mu;\theta_{i})$
can be evaluated at $\theta_{i}^{\pm}$. Since this is not the case
for the bias $b(z)$, one needs to add a further intermediate variable
(see section \ref{sec:Marginalization} below on marginalization).
\item One uses intermediate variables, namely the functions in eqn. \ref{eq:obs-power-basic},
which depend on $\theta_{i}$, and with the help of the chain rule
one can write:
\begin{align}
\left.\frac{\mbox{d}\ln P_{{\rm obs}}\left(\bar{z},k,\mu;\theta_{i}\right)}{\mbox{d}\theta_{i}}\right|_{fid} & =\frac{\partial\ln P_{obs}\left(\bar{z},k,\mu;\theta_{i}\right)}{\partial\ln P_{s}(\bar{z})}\frac{\partial\ln P_{s}(\bar{z})}{\partial\theta_{i}}+\frac{\partial\ln P_{obs}\left(\bar{z},k,\mu;\theta_{i}\right)}{\partial\ln f(\bar{z})}\frac{\partial\ln f(\bar{z})}{\partial\theta_{i}}\nonumber \\
 & +\frac{\partial\ln P_{obs}\left(\bar{z},k,\mu;\theta_{i}\right)}{\partial\ln H(\bar{z})}\frac{\partial\ln H(\bar{z})}{\partial\theta_{i}}+\frac{\partial\ln P_{obs}\left(\bar{z},k,\mu;\theta_{i}\right)}{\partial\ln D_{A}(\bar{z})}\frac{\partial\ln D_{A}(\bar{z})}{\partial\theta_{i}}\nonumber \\
 & +\frac{\partial\ln P_{obs}\left(\bar{z},k,\mu;\theta_{i}\right)}{\partial\ln P(k,\bar{z})}\frac{\partial\ln P(k,\bar{z})}{\partial\theta_{i}}+\frac{\partial\ln P_{obs}\left(\bar{z},k,\mu;\theta_{i}\right)}{\partial\ln b(\bar{z})}\frac{\partial\ln b(\bar{z})}{\partial\theta_{i}}\label{eq:derivatives-of-lnPobs}\\
 & +{\color{red}\frac{\partial\ln P_{obs}\left(\bar{z},k,\mu;\theta_{i}\right)}{\partial k}\frac{\partial k}{\partial\theta_{i}}+\frac{\partial\ln P_{obs}\left(\bar{z},k,\mu;\theta_{i}\right)}{\partial\mu}\frac{\partial\mu}{\partial\theta_{i}}}\nonumber 
\end{align}
where we have ignored the dependence on the damping terms of eqn.
\ref{eq:obs-power-basic} for simplicity. The last two terms are non-vanishing
if one takes the Alcock-Paczynski effect into account for $k$ and
$\mu$, since they are affected by geometrical terms also (see AP-effect
section \ref{sub:The-Alcock-Paczynski-Effect} below).
\end{itemize}
One then has to calculate the intermediate derivatives of $\ln P_{obs}(z,k,\mu;\theta_{i})$
with respect to $P_{s}$, $D_{A}$, $H$, $f$ and $b$ using eqn.
\ref{eq:obs-power-basic}. The logarithm of these quantities is used,
because it simplifies considerably the equations. Notice that the
growth $G(z)$ is inside the definition of $P(k,z)$ so it is not
needed as an intermediate variable, however it can be also used if
wanted. It is important to stress here that the derivatives are calculated
at the \emph{fiducial} value of the parameters, including the fiducial
values of the cosmological functions. The fiducial value of the shot
noise is zero, $P_{s}^{fid}(z)=0$. And under this choice the derivatives
are:

\begin{subequations}

\begin{align}
\frac{\partial\ln P_{obs}\left(\bar{z},k,\mu;\theta_{i}\right)}{\partial\ln P_{s}(\bar{z})} & =\frac{1}{P_{obs}\left(\bar{z},k,\mu;\theta_{i}\right)}\\
\frac{\partial\ln P_{obs}\left(\bar{z},k,\mu;\theta_{i}\right)}{\partial\ln f(\bar{z})} & =\frac{2\beta(\bar{z})\mu^{2}}{1+\beta(\bar{z})\mu^{2}}\\
\frac{\partial\ln P_{obs}\left(\bar{z},k,\mu;\theta_{i}\right)}{\partial\ln b(\bar{z})} & =\frac{2}{1+\beta(\bar{z})\mu^{2}}\\
\frac{\partial\ln P_{obs}\left(\bar{z},k,\mu;\theta_{i}\right)}{\partial\ln H(\bar{z})} & =1\\
\frac{\partial\ln P_{obs}\left(\bar{z},k,\mu;\theta_{i}\right)}{\partial\ln D_{A}(\bar{z})} & =-2\\
\frac{\partial\ln P_{obs}\left(\bar{z},k,\mu;\theta_{i}\right)}{\partial\ln P(k,\bar{z})} & =1
\end{align}
\label{eq: partial-derivs-subeqns}

\end{subequations}

If for the moment and for simplicity we neglect the extra shot noise
contribution $P_{s}$ and the dependence of $k$ and $\mu$ on the
change of cosmological parameters (see AP-effect section below) we
get for each derivative of $P_{obs}(k,\mu,z)$with respect to a cosmological
parameter $\theta_{i}$:

\begin{align}
\left.\frac{\mbox{d}\ln P_{{\rm obs}}\left(\bar{z},k,\mu;\theta_{i}\right)}{\mbox{d}\theta_{i}}\right|_{fid} & =\frac{\partial\ln P(k,\bar{z})}{\partial\theta_{i}}+\frac{\partial\ln H(\bar{z})}{\partial\theta_{i}}-2\frac{\partial\ln D_{A}(\bar{z})}{\partial\theta_{i}}\nonumber \\
 & +\frac{2\beta(\bar{z})\mu^{2}}{1+\beta(\bar{z})\mu^{2}}\frac{\partial\ln f(\bar{z})}{\partial\theta_{i}}+\frac{2}{1+\beta(\bar{z})\mu^{2}}\frac{\partial\ln b(\bar{z})}{\partial\theta_{i}}\label{eq:derivatives-of-lnPobs-simplified-1}
\end{align}


We see that by using intermediate variables, the space of parameters
to be evaluated for the Fisher matrix has been extended by 4 new terms
(they could be more if $P_{obs}$ depends on further cosmological
functions): 

\begin{equation}
\zeta=\{1,1,-2,\frac{2\beta(\bar{z})\mu^{2}}{1+\beta(\bar{z})\mu^{2}},\frac{2}{1+\beta(\bar{z})\mu^{2}}\}\label{eq:dPobsDFuncts}
\end{equation}


The new ``vector'' of derivatives has the form:

\begin{equation}
\Theta=\{\frac{\partial\ln P(k,\bar{z})}{\partial\theta_{i}},1,-2,\frac{2\beta(\bar{z})\mu^{2}}{1+\beta(\bar{z})\mu^{2}},\frac{2}{1+\beta(\bar{z})\mu^{2}}\}\label{eq:bigThetaVector}
\end{equation}
notice that $\theta_{i}$ is a vector also, so that this vector has
a number of components equal to: $\dim(\Theta)=\dim(\theta)+4$.

Since the Fisher matrix is a tensor product of a vector od derivatives,
we can express it as:
\begin{equation}
F_{ij}\approx\frac{\mbox{d}\ln P_{{\rm obs}}}{\mbox{d}\theta_{i}}\frac{\mbox{d}\ln P_{{\rm obs}}}{\mbox{d}\theta_{j}}\approx(\Theta\otimes\Theta)_{ij}\label{eq:FisherTensorProd-1}
\end{equation}


However this is only valid at one single redshift bin and if one wants
to add together Fisher matrices at different redshift bins, one has
to take into account that each $D_{A}(\bar{z}_{i})$, $H(\bar{z}_{i})$,
$f(\bar{z}_{i})$ and $b(\bar{z}_{i})$ is an independent parameter
but we will come to that later. 


\subsubsection{Projecting back to the fundamental cosmological parameters}

If one wishes to express the Fisher matrix just in terms of fundamental
cosmological parameters again, then one needs to perform the variable
transformation back using a so-called Jacobian. This can be seen more
clearly if one expresses eqns. \ref{eq:derivatives-of-lnPobs}, \ref{eq: partial-derivs-subeqns}
and \ref{eq:derivatives-of-lnPobs-simplified-1} as:

\begin{equation}
\frac{\mbox{d}\ln P_{obs}(\mathbf{T}(\boldsymbol{\theta}))}{\mbox{d}\boldsymbol{\theta}_{i}}=\frac{\partial\ln P_{obs}}{\partial\mathbf{T}{}_{j}}\frac{\partial\mathbf{T}_{j}}{\partial\boldsymbol{\theta}_{i}}
\end{equation}
where $\mathbf{T}_{i}=\{\ln P,\,\ln H,\,\ln D_{A},\,\ln f,\,\ln b\}$
and $\partial P_{obs}/\partial\mathbf{T}{}_{i}=\zeta$ from eqn. \ref{eq:dPobsDFuncts}.
The Jacobian is then simply: 

\begin{equation}
J=\frac{\partial\mathbf{T}_{j}}{\partial\boldsymbol{\theta}_{i}}\label{eq:jacobian}
\end{equation}


The Fisher matrix 

\begin{equation}
\tilde{F}_{ab}\approx(\zeta\otimes\zeta)_{ab}\approx\frac{\partial\ln P_{obs}}{\partial\mathbf{T}{}_{a}}\frac{\partial\ln P_{obs}}{\partial\mathbf{T}{}_{b}}
\end{equation}
formed in the intermediate variables $\boldsymbol{T}$ can be transfomed
to the usual one $F$ by calculating:

\begin{equation}
F_{ab}\approx\frac{\mbox{d}\ln P_{obs}}{\mbox{d}\boldsymbol{\theta}{}_{a}}\frac{\mbox{d}\ln P_{obs}}{\mbox{d}\boldsymbol{\theta}{}_{b}}\approx\frac{\partial\ln P_{obs}}{\partial\mathbf{T}{}_{j}}\frac{\partial\mathbf{T}_{j}}{\partial\boldsymbol{\theta}_{a}}\frac{\partial\ln P_{obs}}{\partial\mathbf{T}{}_{j}}\frac{\partial\mathbf{T}_{j}}{\partial\boldsymbol{\theta}_{b}}=J^{T}\,\tilde{F}\,J
\end{equation}


In the praxis however, for calculating the full Fisher matrix, one
interchanges the first element of $\zeta$ by the first element $\partial\boldsymbol{F}_{1}/\partial\boldsymbol{\theta}_{i}$
(corresponding to the derivatives of the power spectrum $P(k,z)$)
since one needs to integrate values in $k$ and $z$. That is why
one uses the vector $\Theta$ from eqn. \ref{eq:bigThetaVector}.
In this case the Jacobian and the $\tilde{F}$ do not look like the
ones presented above. This is however, mathematically equivalent at
this point.


\subsubsection{The Alcock-Paczynski-Effect\label{sub:The-Alcock-Paczynski-Effect}}

The Alcock-Paczynski-effect (AP) reiles on the fact that the scales
$k$ and the direction cosines $\mu$ are changed by geometrical factors
of distance, when the cosmological parameters are changed. This means
that $k$ and $\mu$ depend on $H(z)$ and $D_{A}(z)$ and therefore
are also indirectly functions of the cosmological parameters $\theta_{i}$.
The AP formulas relating the fiducial values of $k_{fid}$ and $\mu_{fid}$
to their transformed values are:

\begin{align}
k_{AP} & =R_{AP}(\mu;\theta)k_{fid}\\
\mu_{AP} & =\frac{H(z;\theta)}{H(z;\theta_{fid})}\frac{\mu_{fid}}{R_{AP}(\mu;\theta)}
\end{align}
where the geometrical function $R_{AP}$ is defined as:

\begin{equation}
R_{AP}(\mu;\theta)=\sqrt{\frac{[D_{A}(z;\theta)H(z;\theta)\mu_{fid}]^{2}-[(D_{A}(z;\theta_{fid})H(z;\theta_{fid})]^{2}(\mu_{fid}-1)}{[D_{A}(z;\theta)H(z;\theta_{fid})]^{2}}}
\end{equation}


The last two terms of eqn. \ref{eq:derivatives-of-lnPobs} come from
the fact that now $k=k(H(z;\theta),D_{A}(z;\theta))$ and $\mu=k(H(z;\theta),D_{A}(z;\theta))$,
so that one has to use the chain rule for the derivative of $P_{obs}(k,\mu,z;\theta)$:

\begin{align}
\frac{\partial\ln P_{obs}\left(\bar{z},k,\mu;H(z);\theta_{i}\right)}{\partial\ln H(z)} & =\frac{\partial\ln P_{obs}\left(\bar{z},k,\mu;\theta_{i}\right)}{\partial\ln H(z)}\\
 & +\frac{\partial\ln P_{obs}\left(\bar{z},k,\mu;\theta_{i}\right)}{\partial\mu}\frac{\partial\mu}{\partial\ln H(z)}+\frac{\partial\ln P_{obs}\left(\bar{z},k,\mu;\theta_{i}\right)}{\partial k}\frac{\partial k}{\partial\ln H(z)}\nonumber 
\end{align}
and similarly for the dependence on $D_{A}(z)$.

Now we can write down explicitly the intermediate derivatives (evaluated
at the fiducial value as usual):

\begin{equation}
\frac{\partial\ln P_{obs}\left(\bar{z},k,\mu;\theta_{i}\right)}{\partial k}=\frac{\partial\ln P\left(\bar{z},k;\theta_{i}\right)}{\partial k}
\end{equation}


\begin{equation}
\frac{\partial\ln P_{obs}\left(\bar{z},k,\mu;\theta_{i}\right)}{\partial\mu}=\frac{4\beta(z)\mu}{1+\beta(z)\mu^{2}}
\end{equation}
and the derivatives of $k$ and $\mu$ with respect to $\ln D_{A}$
and $\ln H$ are:

\begin{subequations}

\begin{align}
\frac{\partial\mu}{\partial\ln D_{A}(z)} & =-\mu(\mu^{2}-1)\\
\frac{\partial\mu}{\partial\ln H(z)} & =-\mu(\mu^{2}-1)\\
\frac{\partial k}{\partial\ln D_{A}(z)} & =k(\mu^{2}-1)\\
\frac{\partial k}{\partial\ln H(z)} & =k\mu^{2}
\end{align}


\label{eq: derivs-of-k-mu-AP}\end{subequations}

The interesting thing about eqns.\ref{eq: derivs-of-k-mu-AP} is that
one can see that when evaluating the derivative of $P_{obs}$ w.r.t
a cosmological parameter, the scales $k$ and the direction cosine
angles $\mu$ get mixed, giving a very powerful probe of cosmology.
Using this, the vectors that form the Fisher matrix (either $\zeta$
or $\Theta$, depending on the choice) have to be extended in their
indices corresponding to $\ln D_{A}$ and $\ln H$. So that they are
now more complicated:

\begin{multline}
\Theta=\left\{ \frac{\partial\ln P(k,\bar{z})}{\partial\theta_{i}},1+\frac{4\beta(z)\mu}{1+\beta(z)\mu^{2}}(-\mu(\mu^{2}-1))+\frac{\partial\ln P\left(\bar{z},k;\theta_{i}\right)}{\partial k}k\mu^{2},\right.\\
\left.-2+\frac{4\beta(z)\mu}{1+\beta(z)\mu^{2}}(-\mu(\mu^{2}-1))+\frac{\partial\ln P\left(\bar{z},k;\theta_{i}\right)}{\partial k}k(\mu^{2}-1),\frac{2\beta(\bar{z})\mu^{2}}{1+\beta(\bar{z})\mu^{2}},\frac{2}{1+\beta(\bar{z})\mu^{2}}\right\} 
\end{multline}


However, using the procedure described above, one can see that the
Fisher matrix will have the same dimensions and that the Jacobian
will be the same as in eqn. \ref{eq:jacobian}.


\subsection{Marginalization\label{sec:Marginalization}}

When using the form of the observed power spectrum presented before,
it is not always possible to ``project back'' the matrix $\tilde{F}$
into the matrix $F$ which contains only fundamental cosmological
parameters. This is the case for the bias or the extra shot noise,
since there we do not know how the derivatives $\partial\ln b(z)/\partial\theta_{i}$
or $\partial P_{s}(z)/\partial\theta_{i}$ look like, since those
are observational and/or phenomenological quantities which do not
yet have a model in terms of fundamental cosmological parameters.
Therefore, these terms will always remain and cannot be expressed
as a function of cosmological parameters and have to be either fixed
or marginalized over. In terms of the Fisher matrix $F_{ij}$, the
marginalization is perfomed easily by removing from the inverse $F_{ij}^{-1}$
all rows and columns corresponding to the index of the quantity to
marginalize. More about this will be explained later on.



\subsection{Validation of the code within the Euclid collaboration}
\begin{itemize}
\item 4 Steps: from the basics to the more complicated
\item Shape parameters, z-obswevables and non-diagonal terms
\item WL: Cases
\end{itemize}

\subsection{Extensions of the Fisher matrix approach to higher orders}
\begin{itemize}
\item Taylor expansion of the Fisher matrix
\item Fisher tensors
\item Variations in parameter space
\end{itemize}






%----------------------------------------------------------------------------------------