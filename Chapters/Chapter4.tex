\chapter{Statistics in Cosmology} % Main chapter title

\label{Statis} % For referencing the chapter elsewhere, use \ref{Chapter1} 

%----------------------------------------------------------------------------------------

\section{Correlation functions and the power spectrum}
\begin{itemize}
\item Gaussian random variables
\item Statistical isotropy and homogeneity
\item Correlation functions and power spectrum
\end{itemize}

\section{Likelihood and the Bayesian approach}
\begin{itemize}
\item Bayes Theorem 
\item Likelihood
\item Priors, posteriors and evidence
\end{itemize}

\section{Fisher Matrix formalism}
\begin{itemize}
\item Approximation of the Gaussian likelihood at the minimum
\item Ellipses and confidence contours
\item Marginalization and maximation
\end{itemize}

\section{The Fisher Matrix Code}
\begin{itemize}
\item This serves as a manual for the Fisher Tools code.
\end{itemize}

\subsection{Methodology of Fisher forecasts}
\begin{itemize}
\item Derivatives of cosmological functions
\item Integration or summation in $k$ or $\ell$ space
\item Alcock-Paczynski effect
\item Redshift binning 
\item Shape, redshift and nuisance parameters
\item Marginalization
\end{itemize}

\subsection{Validation of the code within the Euclid collaboration}
\begin{itemize}
\item 4 Steps: from the basics to the more complicated
\item Shape parameters, z-obswevables and non-diagonal terms
\item WL: Cases
\end{itemize}

\subsection{Extensions of the Fisher matrix approach to higher orders}
\begin{itemize}
\item Taylor expansion of the Fisher matrix
\item Fisher tensors
\item Variations in parameter space
\end{itemize}






%----------------------------------------------------------------------------------------