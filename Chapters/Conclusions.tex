\chapter*{Conclusions \label{Conclu}} % Main chapter title

 % For referencing the chapter elsewhere, use \ref{Chapter1} 


%\todos
%----------------------------------------------------------------------------------------


%----------------------------------------------------------------------------------------

In this thesis we have investigated non-linear structure formation in models of Dark Energy and Modified 
Gravity which go beyond the standard $\lcdm$ scenario. We have focused on predicting its impact
on parameter constraints and contrasting the difference of the linear and the non-linear calculations on observable properties like Weak Lensing, Galaxy Clustering and the evolution of background quantities.

In \cref{chap:DE-MG-Overview} we introduced the theoretical framework of cosmology, which is
based on Einstein's General Relativity (GR) and then we proceeded to explain the concordance $\lcdm$
model and its main properties. 
We have argued that despite the actual data being well explained by the standard model, 
there are some unsatisfactory properties with the Cosmological Constant, namely the so-called fine-tuning
and coincidence problems, that motivate an extension of General Relativity with an extra degree of freedom.
We have focused on theories of Dark Energy (DE) and Modified Gravity (MG) in which this extra degree of freedom  is represented by a dynamical scalar field. The scalar field can not 
only modify the background cosmological solutions,
but in some cases it can also lead to the appearance of an extra "fifth-force" acting between test bodies.
In the second half of \cref{chap:DE-MG-Overview} we have classified the DE and MG models into universally coupled  and non-universally coupled. The former are those models in which all matter and radiation species couple
in the same way to the scalar field, while the latter are models in which just a specific matter component feels the influence of the scalar field. 

General Relativity has been thoroughly tested at laboratory and solar system scales, therefore the coupling to standard matter (baryons) is very well constrained to be negligibly small.
Therefore, universally coupled models have to invoke a "screening" mechanism that recovers GR at small scales (see \cref{sec:universal-coupling}).
On the other hand, non-universally coupled theories pass the stringent solar system constraints by decoupling the baryons
and allowing only for dark sector interactions, involving the scalar field and dark matter or neutrinos, as we discuss in \cref{sec:nonuniversal-coupling}.

For universally coupled theories, we focused on Effective Field Theory (EFT) models and Horndeski models, or on Parameterized Modified Gravity, which
are very general descriptions of theories of GR plus a scalar field.
EFT includes in the action all possible terms allowed by symmetries at first order in perturbation theory, while Horndeski is the most general theory
of GR plus a scalar field, which is second order in the equations of motion and free of ghost instabilities.
Both theories can be mapped to each other at the linear order, by using the so-called $\alpha$-functions.
We explained this more in detail in \cref{sub:EFT-of-DE}.
If we want to study general modifications of gravity affecting the gravitational potentials $\Psi$ and $\Phi$, but we do not want to focus on a particular Lagrangian description, we can parameterize the deviations of GR in terms of two
general functions of scale and time, namely $\mu(k,a)$ and $\eta(k,a)$. As we explain in \cref{sub:parameterizing-MG}, $\mu$ expresses the deviations of the relativistic Poisson equation, while $\eta$
expresses the gravitational anisotropic stress (also known as gravitational slip), which is the ratio $\Phi/\Psi$.
Both functions reduce to unity in the standard GR case or if the DE model just modifies the background equations.

For non-universally coupled theories, we focused on two distinct scenarios. The first one called Coupled Dark Energy
(CDE) is a model in which there is a dark sector interaction between Dark Matter and the Dark Energy field, see \cref{sub:CDE}. This leads to a fifth-force that modifies the growth and the clustering of DM particles and introduces a gravitational bias 
We studied this model in the non-linear regime, based on cosmological N-body simulations and we found
noticeable differences in the non-linear matter power spectrum dependent on the strength of the DM-DE coupling, as detailed in \cref{chap:Fitting-CDE}.
The second non-universally coupled model we study is Growing Neutrino Quintessence (GNQ) in which
the mass of the neutrinos is directly coupled to the scalar field (referred to as ``cosmon``).
In this model Dark Matter and baryons follow standard gravity, but neutrinos feel an extra force among them, which
is very strong (of the order of $10^2$ times the gravitational force) and leads to the formation of large neutrino lumps. Depending on the parameters of the model, these lumps are stable and grow with time
or they present very rapid oscillations, dissolving and forming again. To study the dynamics of these lumps
and their backreaction effect on background quantities, we perform specialized N-body simulations, as we 
detail in \cref{chap:GNQ}. 












