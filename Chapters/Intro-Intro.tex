\chapter*{Introduction \label{IntroIntro}} % Main chapter title
 % For referencing the chapter elsewhere, use \ref{Chapter1} 
\addcontentsline{toc}{chapter}{Introduction}


%----------------------------------------------------------------------------------------


%----------------------------------------------------------------------------------------

%Humans have been wondering about their place and purpose in the Universe since the beginning of history.
%It all started with the curiosity of looking at the stars in the night sky and at the processes of nature around us.
%The processes leading to the formation of the first stars, the first galaxies and then the first solar systems, are very complicated
%and posses a great degree of randomness. 
%There are many processes influencing the formation of these astrophysical objects and only a very 
%precise combination of conditions, allows planets to host life in the way we know it.
%Therefore, it is quite surprising that a species that formed in a small planet on a very typical galaxy, has been capable 
%to develop such an astonishing degree of sophistication in theoretical and observational techniques. 
%We are now able to investigate and explain the Universe from very small subatomic scales to 
%distances of billions of light-years.
%Cosmology is a relatively new science, with less than 100 years of theoretical and observational development.
%At the beginning, it was a very inexact discipline in physics, compared to other more established and observationally tested fields of research, like electromagnetism or statistical mechanics.
\done\todo{Implement Valerias comments}
In the first years after Einstein crafted his theory of General Relativity (GR), other important theoretical and experimental physicists like Hubble, Slipher, Lema\^{\i}tre and Friedmann 
(among many others) delivered the first steps to the discovery 
that the Universe is a dynamical entity and laid the foundations for the field of modern cosmology.
%After that, the advances d

During the 20th century, the advances in cosmology were exceptional. Many more galaxies with many different morphologies were discovered,
and it was found that they were not just randomly located in space, but that they clustered into large and coherent structures. The Cosmic Microwave Background (CMB) radiation was discovered, which gave us a picture of the early Universe and provided the initial conditions for our calculations. 
Later on, galaxy rotation curves and galaxy cluster dynamics hinted strongly at the existence of a cold "Dark Matter" (CDM) component that interacts only gravitationally with common matter.
Finally, towards the end of the century, supernovae observations confirmed that the Universe 
has been experiencing an accelerated rate of expansion in recent times, which might be explained
by introducing a Cosmological Constant ($\Lambda$) into Einstein's equations. 
By the turn of the century, the standard concordance
model of cosmology, known as $\Lambda$-Cold-Dark-Matter ($\lcdm$), was already a well established theory.

In the last decade, cosmology has entered the so-called \emph{precision era}, and it is now a field of science that is driven by large amounts of data; modern measurements are able to constrain the parameters of the cosmological model with very high precision. 
This is possible, despite the narrow window we have for 
observations, compared to other areas of science. In cosmology we only observe basically electromagnetic radiation 
and how its wavelength is redshifted with time and space, plus the positions of galaxies
in the sky and their shapes.
We are now able to perform measurements of very small effects, which
were thought to be too difficult to be realized in practice just a few years ago. An incomplete list of these measurements 
are: Weak (gravitational) Lensing (WL), Baryon Acoustic Oscillations (BAO), Redshift Space Distortions (RSD) and CMB polarization.
Very recently, due to the newly discovered detection of gravitational waves, a new window of gravitational wave astronomy is now open and cosmologists are already thinking on how to use it to constrain, even more, the parameters of the Universe.

In this thesis we will deal with two topics in cosmology that have gained a lot of attention in recent years. The first one is the investigation of the possible extensions to the $\lcdm$ model and the modifications of standard Einstein's General Relativity. The second one is the study of non-linear
formation of large scale structures in the Universe.

Modified Gravity and Dark Energy have earned a lot of interest, since there is yet no successful explanation of the Cosmological Constant problem ---its measured value does not match, by far, the expectations from the Quantum Field Theory point of view. Moreover, it is difficult to explain why the onset of acceleration is happening ``just now`` in cosmological time scales, in other words, why the energy density fraction of Dark Energy and Dark Matter are of the same order of magnitude just during a very short period of time, and that time is precisely now.

Non-linear structure formation is of great importance nowadays, since present and future observations are capable of 
measuring more deeply into the non-linear regime which manifests itself at small scales. 
These scales contain a lot of information about the underlying cosmology and might hint at physics beyond the concordance model. Cosmological many-particle simulations and perturbation theory
have confirmed that these scales contain valuable information, 
but cosmologists have realized that extracting this information in practice is a very difficult task.

In the first half of \cref{chap:DE-MG-Overview}, 
we briefly review the formalism of General Relativity and
the standard cosmological $\lcdm$ scenario. We will explain why the Cosmological 
Constant is not completely satisfactory and we will derive the linearized
Einstein equations, which form the departure point of the theory of 
cosmological structure formation.

In the second half of \cref{chap:DE-MG-Overview} we introduce the concepts of Dark Energy 
and Modified Gravity driven by a scalar field. We will divide the models into those in which the
scalar field is coupled universally to all particles in the Universe and models in which the scalar field is only coupled to specific particles, like Dark Matter or neutrinos. The models for which we will show results in this work are: Coupled Dark Energy, Growing Neutrino Quintessence, Effective Field Theories and Horndeski models. 
We will also deal with parameterizations of Modified Gravity which encompass
general modifications to the relativistic gravitational potentials.

In \cref{chap:Statistics} we explain the underlying concepts in statistics that will help us
make sense of the observations of galaxy surveys and non-linear structures in the Universe.
We detail the Bayesian approach to statistical inference and how we can forecast the results of future 
experiments using the Fisher Matrix formalism. We focus on two observables: Galaxy Clustering
and Weak Lensing. The first one is the study of the two-point correlation function of galaxies
and the second is the correlation among galaxy ellipticities, also known as cosmic shear. 
This chapter includes details on the implementation
of a Fisher Matrix forecasting code developed by the author, which was used in the author's publications.

\Cref{chap:MG-forecasts} studies 
three general parameterizations of Modified Gravity, two of them in which the deviations compared to standard GR are parameterized as smooth functions of time and one in which the deviations are binned in independent redshift intervals. We perform forecasts for future surveys using Galaxy Clustering and Weak Lensing and we test
thoroughly the effects of including non-linear prescriptions into the analysis. We also look at the correlation
between parameters and how we can find a decorrelated set of optimally constrained parameters.

The purpose of the next investigation, detailed on \cref{chap:Fitting-CDE}, is to use fitting functions from N-body simulations in a specific model, namely Coupled Dark Energy, to improve 
previous forecasts on the coupling parameter governing a "fifth-force" interaction between CDM and 
the scalar field. We also show how the ignorance on the correct non-linear power spectrum can bias our results.

In \cref{chap:nonlinear} we elaborate on a technique called "eikonal Renormalized Perturbation Theory"
which is capable of yielding the non-linear matter power spectrum at mildly non-linear scales, by using resummation methods borrowed from Quantum Field Theory. 
We apply this method to a very general theory of gravity plus a scalar field, called Horndeski's theory, but we restrict ourselves to some special limiting cases, in order to  simplify the calculations. This  method
is very promising, because it could provide a faster way of calculating non-linear corrections to the power spectrum for Horndeski models and therefore improve the actual constraints on the model parameters, which are based on linear quantities only.

Finally, in \cref{chap:GNQ} we present a different model of Dark Energy, called Growing Neutrino Quintessence.
In this scenario, the sum of the neutrino masses are varying as a function of time and space, 
and are driven by the value of the Dark Energy scalar field. 
This yields very interesting phenomenological predictions, but also complicates the equations, 
which become highly non-linear. Therefore a non-perturbative treatment is needed, that also takes into 
account the effects of backreaction.
We perform our own (non-Newtonian) N-body simulations and find some interesting regions in parameter space, 
in which the evolution of the cosmological background is very similar to the standard $\lcdm$ scenario, 
but in which neutrinos form very large structures, so-called "lumps".
Interestingly, we discover that the dynamics of these neutrino lumps follow two very distinctive regimes.

Most of the work presented here has been published by the author in different papers. Some of the contents
of \cref{chap:MG-forecasts} have appeared in \cite{casas_linear_2017}. \Cref{chap:Fitting-CDE} is based
on \cite{casas_fitting_2016}, while parts of \cref{chap:GNQ} have been presented in \cite{casas_dynamics_2016-1}.
Certain sections of \cref{chap:Statistics} and \cref{chap:DE-MG-Overview} have also appeared in some of these publications.
The work explained in \cref{chap:nonlinear} and the \textsc{FisherTools} code described in \cref{chap:Statistics}
belong to papers in preparation.


%
%
%In the conclusions chapter we summarize again the main results for each of the projects,
%we connect the concepts together and we give some hints at future work to do.







 