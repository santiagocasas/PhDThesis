% Chapter 1

\chapter{Dark Energy and Modified Gravity \label{Overview}} % Main chapter title

 % For referencing the chapter elsewhere, use \ref{Chapter1} 

%----------------------------------------------------------------------------------------

% Define some commands to keep the formatting separated from the content 
\newcommand{\keyword}[1]{\textbf{#1}}
\newcommand{\tabhead}[1]{\textbf{#1}}
\newcommand{\code}[1]{\texttt{#1}}
\newcommand{\file}[1]{\texttt{\bfseries#1}}
\newcommand{\option}[1]{\texttt{\itshape#1}}

%----------------------------------------------------------------------------------------
%

The theoretical framework to describe gravity and space-time in cosmology is Einstein's General Relativity which is
one of the cornerstones of modern physics.
Einstein formulated this theory more than 100 years ago, without the purpose
of explaining cosmology, but mostly to solve the theoretical challenges posed by his relativistic mechanics under the influence of a gravitational field. Besides maybe the
perihelion of Mercury there was no experimental need for it. It was more than a decade later, thanks to observations
by Hubble and Slipher, that physicists where convinced that there were objects much farther away from
our galaxy and that these objects were receding away from us.

%At this point the field of physical cosmology was born and the theoretical basis for it,
%based on General Relativity, was developed by Lema\^{\i}tre, Friedmann, and Einstein himself among other
%notable scientists.

One century later, General Relativity has passed numerous very stringent tests; from laboratory experiments (\cite{cite}) , to
low orbit tests (\cite{cite}) and solar system tests (\cite{cite}), to pulsar timing tests and the recent
exciting first detection of gravitational waves (\cite{cite}).
It is impressive that a theory that was formulated on the grounds of some very basic
principles, has proven to be so accurate across several orders of magnitude in scales.

To explain the accelerated expansion of the Universe, which was observationally verified almost 20 years ago,
(\cite{cite, supernova, 1998}), Einstein's General Relativity needs to invoke a cosmological constant, which despite the fact that
it is allowed by the theory (see Lovelock's theorem below in\cref{sub:Einstein-Hilbert} ),
possesses many unsatisfactory properties. 
We will review this issue in \cref{sub:CC-problem} below.

In order to provide some context for the main results of this dissertation, we will
provide in the following sections a very brief overview of the standard cosmological model. 
In \cref{sec:GR-framework} we will review the main principles and the mathematical 
formulation of General Relativity, its field equations and
its linearized Newtonian limit.
In \cref{sec:Standard-LCDM} we will deal with the composition of the Universe,
its evolution and the standard cosmological scenario.


\section{The framework of General Relativity \label{sec:GR-framework}}

%For Einstein, Special Relativity (SR) was unsatisfactory and didn't represent a complete theory, because it dealt with 
%intertial frames and under the influence of
%a gravitational field, objects would be accelerated. Moreover due to the equivalence of mass and energy in SR, 
%an object with a high kinetic energy in horizontal direction would have to be accelerated differently towards the Earth, than an object
%with a smaller velocity, therefore violating Newtonian observations, which claim that all test bodies experience
%the same acceleration in a gravitational field, regardless of its velocity or composition.
%This observation that the inertial and gravitational masses must be equivalent was of striking significance for Einstein, 
%which elevated it to a guiding principle in the construction of a relativistic theory of gravity (see \cite{pais, wald, bartelmann}).

%The really ingenious step was connecting this principle with differential geometry. Then, instead of the rigid 
%Euclidean picture, space and time would become a dynamical 4-dimensional structure, described by differential manifolds.
%Gravity would curve this space-time, but in order to maintain the equivalence principle, it is always possible to transform away gravity 
%and map it to a flat Euclidean space, which is precisely one of the properties of a differentiable manifold.
%If one studies the consequences of this simple principle, it leads, without the need of formulating a theory, 
%to concepts like gravitational redshift and gravitational light deflection. 
%Constructing from there a fully-fledged non-linear theory of gravity took Einstein more than 10 years, with the (maybe indirect) help
%of very bright mathematicians like Grossmann, Levi-Civita and Hilbert.

Although Einstein based its construction of General Relativity on considerations of the \emph{Equivalence Principle},  we can say, in more modern terms, that General Relativity is a theory of 
a dynamical tensor field, the metric $g_{\mu \nu}$, which defines the lengths 
of space-time intervals $ds^2 = g_{\mu \nu} dx^{\mu} dx^{\nu}$ and which is invariant under diffeomorphisms. 
All particles and fields couple to the metric $g$ in a universal way.
This means that the equations of motion and all the physical properties do not depend on the chosen coordinates.
Diffeomorphism invariance is a very important symmetry that has to be respected if one wishes to construct
extensions of gravity. 
%However, at the smallest (Planck length) and largest (super-horizon) scales, there are suggestions
%that this symmetry might be broken in order to be able to construct a consistent theory of Quantum Gravity
%(\cite{cite Rovelli}).



\subsection{The Einstein-Hilbert action and the field equations \label{sub:Einstein-Hilbert}}

Although Einstein postulated the field equations of General Relativity in a heuristic
form, they can be obtained by varying the so-called Einstein-Hilbert action
\begin{equation}\label{eq:Einstein-Hilbert action}
S = \frac{1}{16 \pi G} \int \textrm{d}x^4 \sqrt{-g}\left( R - 2\Lambda + \mathcal{L}_m \right) \quad 
\end{equation}
with respect to the metric $g_{\mu \nu}$.
Here, $R=R^\mu_\nu$ is the Ricci scalar and $R^\munu$ is the Riemann tensor (see standard GR textbooks like \cite{wald} for its definition), 
which are functions of derivatives of the metric. The volume element is defined as $\textrm{d}x^4 \sqrt{-g}$, where
$\sqrt{-g}$ is the square root of the determinant of the metric.
Furthermore, the cosmological constant is $\Lambda$, the gravitational constant is $G$ and $\mathcal{L}_m $ is the Lagrangian of matter and radiation species.
Then the variation ($\delta S / \delta g_{\mu \nu} = 0$), yields the field equations:
\begin{equation}\label{eq:Einstein-field-equations}
G_\munu + g_{\mu \nu} \Lambda = 8 \pi G T_{\mu \nu} \quad ,
\end{equation}
where the Einstein tensor is defined as $G_\munu \equiv R_{\mu \nu} - \frac{1}{2}g_{\mu \nu} R \;$ and the 
energy-momentum tensor is defined as:
\beeqp$
T_\munu = -\frac{2}{\sqrt{-g}} \frac{\delta \mathcal{L}_m}{\delta g_\munu}
$
What \cref{eq:Einstein-field-equations} expresses is that geometry (and therefore the dynamics of the metric) is sourced by 
the energy and momentum of the fields living on this manifold.
As it was famously expressed by \cite{Misner, Wheeler, Gravitation}:
"Matter tells space-time how to curve and space-time tells geometry how to move" .

Another important property of these field equations, is that due to a purely geometric property called the Bianchi identity, which
states that the covariant divergence of the Einstein tensor is identically zero:
$\nabla_{\mu} G^\munu = 0$, 
we can ensure that the energy-momentum tensor is locally conserved:
\beeqc$
\nabla_{\mu} T^\munu = 0
$
therefore one recovers all the well-known properties of classical and fluid mechanics.

%Einstein's field equations are very complicated to solve analytically in its full glory. They consist on
%10 coupled non-linear partial differential equations.
%There are of course many interesting properties, solutions and implications of these equations, 
%which are out of the scope of this work.
%We refer the interested reader to very good textbooks like \cite{Carroll, Schultz, Misner, Thorne, Wheeler, Wald}.
%The richness and complexity of this theory is one of the reasons it has to be so successful explaining 
%star formation, gravitational collapse, black holes, gravitational waves and the evolution of the Universe.

%\subsubsection{Lovelock's theorem and the uniqueness of General Relativity}

One might then ask if these field equations are unique, especially if as in our case, 
we are interested in testing these equations at the very largest scales of the Universe and we might 
be interested in modifying General Relativity in order to match current observations.
Thanks to a theorem by Vermeil and Cartan (\cite{cite Vermeil, Cartan, 1921}) 
and further simplified by Lovelock (\cite{1970, Lovelock}) we can state that in 4 dimensions,
the only divergence-free, rank-2 tensor $\mathcal{G}$, which depends on at most second derivatives
of the metric must be of the form:
\beeqp$
\mathcal{G} =  \alpha R_{\mu \nu} + \left( \Lambda - \frac{\alpha}{2} R \right) g_{\mu \nu}
$
Therefore, to ensure the correct Newtonian limit of the theory (Newtonian Poisson equation), we must set $\alpha \equiv 1$
and the proportionality constant between the Einstein tensor $G_\munu$  and the energy-momentum tensor
$T_\munu$ has to be set to $8 \pi G$.
This theorem has fundamental implications for cosmology, which we will comment 
in the following sections, when needed.
%First of all, it states that the Cosmological Constant (CC) has to appear in the classical theory of gravity,
%therefore, if ---as we will see below in \cref{sub:CC-problem})--- we are not satisfied with the CC as an explanation
%of the accelerated expansion of the Universe, we still have to explain why $\Lambda$ disappears from the Einstein's field equations.
%Second, this theorem gives a sort of roadmap to look for modifications of gravity, which can account for the late-time expansion of the Universe.
%One option is to go beyond 4 dimensions (as in DGP models \cite{DGP models}),
%violate the conservation of the energy-momentum tensor with some exotic species or
%add extra degrees of freedom, since with the metric alone we can't construct a more general theory.
%This last option has led to the recent interest in modified gravity theories with a scalar field (see \cite{cite some scalar field}), 
%with vector fields (see \cite{cite}) and with the addition of several metric (tensor) fields (see \cite{cite}). 


%----------------------------------------------------------------------------------------

\section{The standard cosmological model \label{sec:Standard-LCDM}}

The cosmological principle states
that no observer in the Universe is special and that each observer sees the Universe in the same way independently of
spatial rotations, therefore space-time has to be described by a homogeneous and isotropic metric $g_\munu$.
If furthermore, we can define a foliation of space-time, in which there is a preferred timelike direction, orthogonal
to the spatial hypersurfaces, we end up with a metric which solves the Einstein's field equations \ref{eq:Einstein-field-equations};
the so-called Friedmann-Lema\^{\i}tre-Robertson-Walker (FLRW) metric:
\beeqc$\label{eq:FLRW-full}
ds^2 = g_\munu dx^\mu dx^\nu = -dt^2 + a^2 (t) d\varsigma^2 
$
where $a$ is the scale factor, $t$ the cosmic time coordinate, and $d\varsigma^2$ is the time-independent spatial metric:
\beeqc$\label{eq:FLRW-spatial}
d\varsigma^2  = \gamma_{i j} dx^i dx^j = \frac{dr^2}{1-kr^2} + r^2(d\theta^2 + \sin^2 \theta d \phi^2)
$ 
where $r$ is the radial coordinate, $\theta$ the polar angle and $\phi$ the azimuthal angle.
The curvature $k$ which can be 0, 1 or -1, corresponds to Universes which are flat, closed or open, respectively.
Due to the stringent constraints on $k$ given by recent observations,
we will use for the rest of this work only a flat geometry with $k=0$.
Also it is standard convention that "Greek" indices run from 0 to 3, as in \cref{eq:FLRW-full},
while "Latin" indices run from 1 to 3 as in equations involving only spatial coordinates, like \cref{eq:FLRW-spatial}.


\subsection{The Friedmann equations \label{sub:Friedmann-eqs}}

The FLRW metric \cref{eq:FLRW-full} due to its scale factor, which is dependent on time, implies immediately that the Universe can expand
in its spatial coordinates. The equations describing the evolution
of the scale factor are called the \emph{Friedmann} equations. To derive them,
we need to introduce in the right hand side of Einstein's equations \cref{eq:Einstein-field-equations} an energy-momentum tensor of a perfect fluid:
\beeqc$
T^\munu = (\rho + p)u^\mu u^\nu + p g^\munu
$
which is justified since at cosmological scales, we expect the background matter in the Universe to be absent of dissipative and viscous forces.
The density $\rho$ and the pressure $p$ are the sum of the densities and pressure terms of all matter and radiation species in the universe.
If we write down now the (00) and ($i i$) components of the Einstein's field equations \cref{eq:Einstein-field-equations}, computing
the Riemann and Ricci tensors of a FLRW metric, 
we end up with two ordinary differential equations for the scale factor $a(t)$:
\beeqal$
\frac{\dot a}{a} &= \frac{8 \pi G}{3} \rho + \frac{\Lambda}{3}  \label{eq:Friedmann-1st}\\
\frac{\ddot a }{a} &= -\frac{4 \pi G}{3} (\rho +3 p) \frac{\Lambda}{3} \label{eq:Friedmann-2nd}
$
These are the so-called Friedmann equations, which define the evolution of the Hubble function defined as:
\beeqp$
H(t) \equiv \frac{\dot a(t)}{a(t)}
$
The critical density of the Universe is defined as:
\beeqc$
\rho_{cr} = \frac{3 H^2}{8 \pi G}
$
which is the critical density that an Universe with zero curvature $k=0$ would have according to the observed value of the Hubble function.
So, for each species in the Universe, with energy density $\rho_i$, we can define the energy density fraction as:
\beeqc$
\Omega_i (t) = \frac{\rho_i (t)}{\rho_cr (t)}
$ 
so that the first Friedmann equation \cref{eq:Friedmann-1st} (for a flat Universe) can be written as:
\beeqp$
\sum_i \Omega_i (t) = 1
$
Another important quantity for each matter species is its equation of state:
\beeqc$
w \equiv \frac{p}{\rho}
$
which enters in its cosmological evolution equation.


\subsection{The $\mathbf{\Lambda}$CDM model \label{sub:LCDM}}

As we have seen before, the field of cosmology is relatively new, with less than 100 years of theoretical development
and even much less time of observational progress. 
However, in the last few decades, with the precise measurements of the Cosmic Microwave Background (CMB) radiation,
the rapid progress in galaxy surveys and the impressive development of cosmological simulations, an observationally very successful standard model has emerged, the so-called
$\lcdm$ model.

In this model, and according to the latest observations (see \cite{planck_collaboration_planck_2016-1}),
almost 70\% of the energy density of the Universe is composed by vacuum energy, attributed to the Cosmological Constant
$\Lambda$, 25\% by Cold Dark Matter and less than 5\% by baryons. The remaining components are photons ---which are basically negligible today, despite the amount of light and radiation in the Universe--- and massive neutrinos,
whose mass and therefore its contribution to the "cosmic pie" has not been measured precisely enough yet.
%Up to 1965?, with the discovery of the Cosmic Microwave Background (CMB) radiation by \cite{Penzias and Wilson},
%cosmology was a mere speculative science. 
%Only with the confirmation of the "Hot Big Bang paradigm" by the measurement of the radiation left over
%380,000 years after the Big Bang, is that physicists started to make precise descriptions about the composition and 
%the evolution of the matter species composing the Universe. The first CMB observations clarified that the Universe started
%as a hot plasma of electrons, protons and photons, whose interactions in an expanding space-time matched very well the observed properties
%of the present Universe.

%However, there was always a piece missing in the puzzle, since the observed densities of visible matter in the Universe were too low.
%For many years, there were big discussions about open and closed Universes and about invisible forms of matter.
%Only with the amazing development of galaxy surveys in the 1980's and precise measurements of stars velocities in the Milky Way and other galaxies
%and the development of cosmological N-body simulations in the 1990's, most cosmologists were convinced that there had to be a form of matter in the Universe
%that does not interact with baryons or with electromagnetic fields, but it interacts in a standard way with gravitational fields (\cite{cite;
%recent peebles, Dolag DM review, reviews, books}).
%They called it "Cold Dark Matter" (CDM) because of its apparent vanishing pressure and its preference for clustering into big structures.
%
%However, there were still some inconsistencies between the growth of structures in a pure 'CDM+baryons' model, predicted
%by simulations and the observed structures in the Universe (see \cite{reviews Dolag, peacock}), so that many scientists were claiming for the addition
%of a cosmological constant. However, this claims were not significant until in 1998 the teams led by \cite{Perlmutter and ..} used Supernova Type Ia
%observations to prove that the Universe was experiencing a phase of accelerated expansion. This led to the re-introduction of the Cosmological Constant
%and due to the theoretical problems posed by its so small observed value, to the development of a whole new 
%field of research: Modified Gravity and Dark Energy.

%This very rough and quick overview explains the name of the current standard cosmological model: the $\lcdm$ model. 
This special mixture of cosmological ingredients in the Universe, leads to a well defined background evolution of the Hubble function $H(z)$. However, as we will see below, the Cosmological Constant is not very satisfactory from the theoretical point of view.
Therefore if we leave open the possibility that the accelerated expansion of the Universe is caused by a "Dark Energy" component, with 
an unknown equation of state $w_{DE}(z)$ (where $w_{DE}(z) = -1$ would correspond to the Cosmological Constant), then together with the "standard" matter species (with $w=0$ for CDM and baryons and $w=1/3$ for radiation), we can write down
the evolution of the Hubble parameter as a function of redshift $z \equiv 1/a - 1 $ :
\beeqalsp$
H^2 (z) =& H_0 \left(  \Omega_c (1+z)^3 + \Omega_b (1+z)^3  + \Omega_r (1+z)^4   
\phantom{\exp \left[ \int_{0}^{z}  \frac{3}{1} \right]} \right. \\
      & \left.  +\Omega_{DE}  \exp \left[ \int_{0}^{z} \dx{}{\tilde z} \frac{3(1+w_{DE}(\tilde z))}{1+\tilde z} \right]     \right)
$
where we have left out neutrinos, since their evolution at high and low redshifts is not so straightforward to write, due to their changing from relativistic to non-relativistic particles in the history of the Universe.
If we assume a constant $w$ and a single matter species, one can analytically solve 
\crefrange{eq:Friedmann-1st}{eq:Friedmann-2nd} and find that the scale factor behaves 
as:
\beeqp$
a \propto t^{\frac{2}{3(1+w)}}
$
Since each species evolves with a different power of the scale factor, there 
were three different epochs in the evolution of the Universe, the radiation dominated era (RDE) in which $a \propto t^{1/2}$, the matter domination era
(MDE) in which $a \propto t^{2/3}$ (also called the Einstein-de Sitter Universe) and finally the future dark energy dominated era, also called the de Sitter regime, in which
the total energy density is constant and therefore $a \propto \exp(H t)$.
From the Hubble function, one can obtain measurable 
cosmological distances to astrophysical objects and the
the age and the size of the observable Universe.

Since the Universe is expanding and we measure astrophysical objects using mainly electromagnetic radiation (except after 2016 \cite{cite LIGO}, 
where Gravitational Wave Astronomy was born), it is important to define certain distances that can be measured in cosmological observations.
Light follows null-geodesics, so that under an FLRW metric as in \cref{eq:FLRW-full}, 
light satisfies the equation:
\beeqc$
- c dt^2 + a^2 (t) d\varsigma^2 = 0 
$
where we have recovered the speed of light $c$ to avoid confusion.
Solving for $\varsigma$ and therefore integrating this equation, defines the comoving distance $d_c$:
\beeqp$
d_c \equiv \int_0^{\varsigma_1} \mathrm{d}\varsigma = - \int_{t_0}^{t^1} \frac{c}{a(t)} \dx{}{t} 
$
Since $H = (\mathrm{d}a/\mathrm{d}t)/a = (\mathrm{d}(1/(1+z))/\mathrm{d}t)/(1/(1+z)) $, then $\mathrm{d} t = -\mathrm{d}z / ((1+z) H) $,
so that the comoving distance can be defined as:
\beeqp$
d_c (z) = \int_{0}^{z} \frac{\dx{}{\tilde z}}{H(\tilde z)}
$
Making similar geometrical considerations (which we will not detail here, see \cite{amendola_dark_2010}), one can find the
luminosity distance:
\beeqc$
d_L = (1+z) d_c
$
and the angular diameter distance:
\beeqc$
d_A = \frac{d_c}{1+z}
$
where we have set the curvature $k$ a priori to zero.


\subsection{The cosmological constant problem \label{sub:CC-problem}}

There are two main problems with the $\lcdm$ model: The first one is $\Lambda$; the second
one, CDM.
We will not deal here with the CDM problem, but it is just worth to say, 
that despite large efforts to find 
the CDM particle, nothing significant has been found so far.

On the contrary, the Cosmological Constant (CC) problem is a more profound one, since it 
involves a missing understanding of both Quantum Field Theory and General Relativity, as we will see now in more detail.
From \cref{eq:Friedmann-1st}, we see that under dark energy domination, 
the CC is of the order of the square of the Hubble parameter today:
\beeqp$
\Lambda \approx H_0^2 = (2.1 h \times 10^{-42} \mathrm{GeV})^2
$
so that as an energy density $\rho_{\Lambda} = \Lambda / 8\pi G$ we would obtain:
\beeqc$
\rho_{\Lambda} \approx 10^{-47} \mathrm{GeV}^4
$
where we have used $1/G = m_{Pl} $ and the Planck mass is equal to $10^{19} \mathrm{GeV}$.
%If we suppose that this constant comes from the vacuum fluctuations of empty space,
%which would be the most logical candidate for an energy theory

\textcolor{red}{\textsc{Section still Incomplete}}


\subsection{\revtext{Gravitational potentials}}


In linear perturbation theory, scalar, vector and tensor perturbations
do not mix, which allows us to consider only the scalar perturbations of the metric. 
We work in the conformal Newtonian gauge, with the
line element given by 
\begin{equation}
ds^{2}=-(1+2\Psi)dt^{2}+a^{2}(1-2\Phi)dx^{2}\,\,\,.
\end{equation}
The potentials $\Phi$ and $\Psi$ are in functions of time and in our notation, they coincide
with the gauge-invariant Bardeen potentials in the Newtonian gauge.


\textcolor{red}{\textsc{Section still Incomplete}}
%\begin{itemize}
%	\item Gravitational potentials $\Phi$ and $\Psi$.
%%	\item Very simple explanation of Gauge choice \comm{probably don't need it}
%	\item Newtonian Gauge
%%	\item Smallness of gravitational potentials \comm{what is this?}
%	\item Remark for later that in GR $\Psi=\Phi$
%	\item Weyl potential (lensing potential, null geodesics)
%	\item{The Newtonian limit}
%\end{itemize}



\chapter{Dark Energy and Modified Gravity} % Main chapter title

\label{DE-MG} % For referencing the chapter elsewhere

%----------------------------------------------------------------------------------------


%----------------------------------------------------------------------------------------

In the introduction to this work we have motivated why an extension of General Relativity
is needed in the light of present observations and an absence of a satisfactory
solution to the cosmological constant problem.
In this chapter we will deal with the most common solution to the Dark Energy problem,
namely the addition of an extra dynamical degree of freedom in the form of a scalar field.
These models are generally called scalar-tensor (ST) or Modified Gravity (MG) theories. 
The term scalar-tensor theories stands for theories in which a scalar field is coupled either minimally or non-minimally
to the metric tensor of gravity. On the other hand, the term 
modified gravity can encompass these but also other much more exotic modifications of GR, 
for example involving extra dimensions and other 
vector or tensor fields \cite{cite, bigravity, LH vector theory, dgp}. 

One can regard this extra degree of freedom as an exotic new form of matter, affecting the 
r.h.s. of the Einstein equation \ref{einstein} and generally coupled in distinct ways to different matter fields, 
or one can consider gravity to have an extra degree of freedom coupled non-minimally to the 
Ricci scalar and therefore modifying 
directly the left hand side of Einstein's field equations. The latter frame of reference
is called the Jordan frame, while the former is usually called the Einstein frame.

For theories in which all matter fields are coupled in the same way to the scalar field, both descriptions
have to be equivalent and we call them universally coupled theories. 
For theories in which the coupling is specific to a matter field (i.e. neutrinos), the theory can in general be only formulated
in one description. These are the so-called non-universally coupled theories.
In the following, we will use this classification to discuss the models studied in this work.
In the first section we will discuss universally coupled theories, which include Quintessence, f(R),
scalar-tensor theories and the Horndeski theory. We will also discuss
more general modifications of gravity which are based on the Effective Field Theory (EFT) approach to
Dark Energy and parameterizations of modifications of gravity, which
are not connected to any particular model and can accommodate any deviation from standard GR at least at the linear 
level in perturbation theory.

In the second section, we will deal with two specific non-universally coupled models of dark energy used in this work. 
The first one,
called Coupled Dark Energy (CDE) which only couples the DE field to dark matter particles and leaves baryon uncoupled, following standard
gravity.
The second one called Growing Neutrino Quintessence (GNQ) in which the neutrino mass is coupled to the DE scalar field, while 
all other species remain uncoupled.
We will explain the motivations for each of these models and its differences in predictions for structure formation and the background evolution of the 
Universe.
 
\section{The Einstein and the Jordan frames}

In the `Einstein-frame` the gravitational part of the action looks standard, but the coupled
particles do not follow the geodesics given by GR, since the e
but 

\section{Universal coupling to matter}

\begin{itemize}
\item define Jordan and Einstein frame \comm{for universal couplings these coincide; you should move this to non-universal coupling part}
\item write action in both cases
\item write that in Universal coupling matter couples to the full $\sqrt{-g}$
\end{itemize}

\subsection{Quintessence}

\begin{itemize}
\item small introduction to quintessence, Ratra, Peebles, Wetterich, Luca
\item different potentials, Brans-Dicke, Exponential potential
\end{itemize}

\subsection{f(R) Theories}
\comm{delete this paragraph}
\begin{itemize}
\item small intro to f(R), Amendola, Felice, Tsujikawa
\item write f(R) as conformal transformation of quintessence (Valeria's paper)
\end{itemize}

\subsection{Horndeski Theory}
\comm{not Horndeski, unless you have results for it; call it 'parameterizing Modified Gravity' and refer to the paper inside; don't refer to QS limit in this chapter but only in the point where you actually make this assumption}
\begin{itemize}
\item Most general single scalar field theory, second order equations of
motion, 4 dimensions, ghost-free
\item 5 Lagrangians which can be parametrized with the $\alpha$ functions
that give more physical insight.
\item Quasistatic limit yields modifications of gravity that depend only
on time and $k^{2}$.
\end{itemize}

In a homogeneous and isotropic universe, the distance element
assumes the form: 
\begin{equation}
ds^{2}=a(\tau)[-(1+2\Psi)d\tau^{2}+(1-2\Phi)\delta_{ij}dx^{i}dx^{j}]\,,
\end{equation}
where the conformal time $\tau$ is related to the cosmic time $t$
by $d\tau=d t/a(\tau)$ and $\Psi$ and $\Phi$ are the two scalar
gauge invariant gravitational potentials, related (modulo sign) to
the Bardeen potentials.

Horndeski models include (almost) all theories
with second order equations of motion in the fields and can be generically
described with five functions of time. They include a variety of particular
models studied in literature, characterised by a universal coupling
to matter fields.

The Lagrangian can be written as ...

where ... is the matter Lagrangian and ... are arbitrary functions
of the scalar field $\phi$, whose canonical kinetic term is $X=-\frac{1}{2}\phi^{;\mu}\phi_{;\mu}$;
the latter is also coupled to the Ricci scalar and the Einstein tensor.

In the following, we will adopt the conformal Newtonian gauge in the
notation of \citep{MaBertschinger}, so that the gauge invariant potentials
$\Psi$ and $\Phi$ are also equal to the gravitational potentials
in the Newtonian limit. In this formulation, the cold dark matter
(CDM, c) perturbation equations for the density contrast $\delta$
and the velocity divergence $\theta$ can be written in Fourier space
in the usual way: 
\begin{align}
\dot{\delta}(\tau,\vk k) & =-(\theta(\tau,\vk k)-3\Phi(\tau,\vk k))\,\,,\\
\dot{\theta}(\tau,\vk k) & =-H\theta(\tau,\vk k)+k^{2}\Psi(\tau,\vk k)\,\,.
\end{align}
Derivating the first of the above equations with respect to $\tau$
and eliminating $\dot{\theta}$ and $\theta$ using both equations,
one obtains the usual second order equation in time for the density
contrast: 
\begin{equation}
\ddot{\delta}(\tau,\vk k)=-(H\dot{\delta}(\tau,\vk k)-3H\dot{\Phi}(\tau,\vk k)+k\text{\texttwosuperior}\Psi(\tau,\vk k)-3\ddot{\Phi}(\tau,\vk k))\,\,.\label{eq:deltadotdot}
\end{equation}

%For dimensional reasons \comment{VP: what do you mean? dimensional
%reasons cannot tell you anything about the amplitude of any quantity}
%and since we expect that the potentials vary only strongly \comment{VP:
%what does it mean to vary only strongly? only or strongly?} at scales
%compared to the horizon $H^{-1}$, we must have that \comment{VP:
%do you mean you are assuming it?} $\ddot{\Phi}\sim H\dot{\Phi}\sim H^{2}\Phi$
%and besides we also expect $\Psi\sim\Phi$ at least at the linear
%level (\comment{VP: not sure how this can be true in MG theories
%nor why we need it at all. I would delete. In general only use precise
%and motivated statements or don't say it at all.}). If we then divide
%Eq. \ref{eq:deltadotdot} by $H^{2}$, the potentials $\Phi$ can
%be neglected with respect to the term containing $k^{2}/H^{2}$, since
%for subhorizon scales ($k\gtrsim0.01h/\mbox{Mpc})$ it is many orders
%of magnitude bigger. Therefore we obtain: \comment{VP: I would completely
%delete all this text after eq.(4) and just say: in the Newtonian limit
%(or directly quasi-static), for which $k/H\ll1$, Eq. 4 reduces to:}
In the following, we will stay within the quasi-static
limit, i.e restrict to scales much smaller than the cosmological horizon
($k/aH\gg1$) and well inside the Jeans length of the scalar field
$c_{s}k\gg1$, so that terms containing $k$ dominate over terms containing
time derivatives. In this case, Eq. \ref{eq:deltadotdot} reduces
to: 
\begin{equation}
\ddot{\delta}(\tau,\vk k)+H\dot{\delta}(\tau,\vk k)+k^{2}\Psi(\tau,\vk k)=0\,.\label{eq:growth-rate-Psi}
\end{equation}
The gravitational potential is then obtained from the Poisson equation,
that relates the gravitational potential to matter and relativistic
species in the universe, by perturbing the Einstein field equations
to first order.

In Horndeski theories, in the quasi-static limit, the deviation
of the gravitational potentials from General Relativity has a known
scale dependence in the Poisson equation, given by \citep{...}: 
\begin{equation}
\mu(k,a)\equiv-\frac{2k^{2}\Psi}{3\Omega_{m}\delta_{m}}=h_{1}\left(\frac{1+(k/k_{*})^{2}h_{5}}{1+(k/k_{*})^{2}h_{3}}\right)\label{eq:Yfunc}
\end{equation}
where $h_{1},h_{3},h_{5}$ are functions of time only. A similar expression,
with different time dependent coefficients $h_{2},h_{4}$, holds for
the gravitational slip $\eta$: 
\begin{equation}
\eta(k,a)\equiv-\frac{\Phi}{\Psi}=h_{2}\left(\frac{1+(k/k_{*})^{2}h_{4}}{1+(k/k_{*})^{2}h_{5}}\right)\,.\label{eq:etaFunc}
\end{equation}
In this work we will consider all $h_{i}$ functions as constants,
for simplicity, although in general they are all functions of time.
%since we are expecting them not to affect so much the evolution of growth with respect to%a fiducial LCDM model. %Once a specific model of the Horndeski class%is chosen, the $h_i$ functions and their time dependence can be calculated%and its approximated constant value during the time of structure growth%can be estimated. The
$k_{*}$ is an arbitrary pivot scale, chosen at large scales close
to the horizon, in order to fulfil the quasi-static limit.

As it can be seen from eqns. \ref{eq:Yfunc} and \ref{eq:etaFunc},
the functions $h_{5}$ and $h_{3}$ are degenerate as well as $h_{4}$
and $h_{5}$. %Since $h_{5}$ appears in the denominator as well as%in the numerator of both equations, a combination of both such as%$\mu(1+\eta)$ also retains the smae structure. %This is of importance%since such a combination would appear in the modified weak lensing%potential. The
$\Lambda\textrm{CDM}$ case is recovered when $\mu=\eta=1$, which
implies $h_{5}=h_{4}=h_{3}$ and an amplitude $h_{1}=h_{2}=1$.

In the following, we will focus on CDM perturbations and deal only
with the function $\mu(k,a)=\mu(k)$ (although perturbations require
two functions of time and scale to fully specify the model). In particular,
we will test the following cases: 
\begin{enumerate}
\item Scale independent $\mu$ with modified amplitude: $h_{5}=h_{3}$ and
$h_{1}>1$ or $h_{1}<1$. 
\item Scale dependent $\mu$ with unity amplitude: $h_{1}=1$ and $h_{5}>h_{3}$
or $h_{5}<h_{3}$. 
\item Scale dependent $\mu$ with modified amplitude: $h_{1}\neq1$ and
$h_{5}>h_{3}\mbox{ or }h_{5}<h_{3}$. 
\end{enumerate}


In logarithmic space $\mu(k)$ is very close to a step function, as
shown in figure \ref{fig:Y(k)-function}, so that we will only use
values for the $h_{i}$ in which the difference of the minimum and
the maximum is of the order of 15\%, since otherwise we would impose
a rather unrealistic structure formation.


\subsection{Effective Field Theory of Dark Energy}
\comm{delete unless there is any result you have on EFT}
\begin{itemize}
\item Another approach to construct a general model for dark energy based
on the allowed symmetries in the action.
\item Can be parametrized with the $\alpha$ functions.
\item Allows in a simple way to create theories beyond Horndeski and to
accomodate non-universal couplings.
\end{itemize}

\section{Non-universal coupling}

\begin{itemize}
\item Motivate dark energy, the coincidence problem and a dark sector interaction
\item Explain why baryons should be uncoupled (Solar system constraints)
\end{itemize}

\subsection{Coupled Dark Energy}

The background evolution for the coupled DE scenario model is described
by the following equations, in which the subscripts $r$, $b$, $c$
and $\phi$, indicate radiation, baryons, cold dark matter (CDM) and
the dark energy scalar field, respectively:

\begin{eqnarray}
\ddot{\phi}+3H\dot{\phi}+\frac{dV}{d\phi} & = & \sqrt{\frac{2}{3}}\beta(\phi)\frac{\rho_{c}}{M_{Pl}}\,\,,\label{eq:quint-kleingordon}\\
\dot{\rho}_{c}+3H\rho_{c} & = & -\sqrt{\frac{2}{3}}\beta(\phi)\frac{\rho_{c}\dot{\phi}}{M_{Pl}}\,\,,\label{eq:cdm-back-density}\\
\dot{\rho}_{b}+3H\rho_{b} & = & 0\,\,,\\
\dot{\rho}_{r}+4H\rho_{r} & = & 0\,\,,\\
3H^{2} & = & \frac{1}{M_{Pl}^{2}}(\rho_{b}+\rho_{c}+\rho_{r}+\rho_{\phi})\,\,.
\end{eqnarray}


We express from now on the scalar field $\phi$ in units of the Planck
mass $M_{pl}\equiv1/\sqrt{8\pi G}$, and choose as potential $V(\phi)$
an exponential $V(\phi)=Ae^{-\alpha\phi}$ \citep{Lucchin_Matarrese_1984,Wetterich_1988}.
The coupling function $\beta(\phi)$ defines the strength of the interaction
between the DE fluid and CDM particles and in the present work we
will restrict our analysis to the simplified case of a constant coupling
$\beta(\phi)=\beta$, although in general it could be a field-dependent
quantity \cite{Amendola_2004,Baldi_2011a}.

The current constraints on a coupling to ordinary matter are very
tight. The ``post-Einstein'' coupling parameter $\bar{\gamma}$
that measures the local admixture of a spin-0 field to gravity is
constrained in Solar System experiments (see e.g. the PDG review \cite{Agashe:2014kda}
and also \citep{Will_2005,Bertotti_Iess_Tortora_2003}) roughly to
$|\bar{\gamma}|\le4\cdot10^{-5}$ . This parameter enters the modification
of the effective Newton constant as $G_{eff}=G_{N}(1-\bar{\gamma}/2)$;
in our notation, this is $G_{eff}=G_{N}(1+4\beta^{2}/3)$ (see below)
and therefore $\beta^{2}=-3\bar{\gamma}/8$. A coupling $\beta_{baryons}^{2}$
appears then constrained to be smaller than $10^{-5}$ roughly, and
we assume therefore that is completely negligible. As a consequence,
baryons follow the usual geodesics of a FLRW cosmology, which allows
coupled DE to pass the stringent local gravity constraints without
the need to employ any screening mechanism \citep{2015arXiv150203888H}.

Due to the exchange of energy between DE and CDM, the energy density
of the latter will no longer scale as the cosmic volume, and by assuming
the conservation of the CDM particle number one can derive the time
evolution of the CDM particle mass by integrating Eq.(\ref{eq:cdm-back-density})
between the present time ($z=0$) and any other redshift $z$:

\begin{equation}
m_{c}(z)=m_{c,0}e^{-\beta(\phi(z)-\phi(0))}\,\,.
\end{equation}


\subsection{Growing Neutrino Quintessence}

Growing neutrino quintessence \cite{amendola_growing_2008,wetterich_growing_2007}
explains the end of a cosmological scaling solution (in which dark
energy scales as the dominant background) and the subsequent transition
to a dark energy dominated era by the growing mass of neutrinos, induced
by the change of the value of the cosmon field which is responsible
for dynamical dark energy. The dependence of the mass of neutrinos
on the cosmon (dark energy) field $\phi$, 
\begin{equation}
m_{\nu}=m_{\nu}(\phi)\propto\hat{m}_{\nu}e^{-\int\beta(\phi)d\phi}\,\,,\beta(\phi)=-\frac{\partial\ln m_{\nu}(\phi)}{\partial\phi}\label{eq: mass_def}
\end{equation}
involves the cosmon-neutrino coupling $\beta(\phi)$ which measures
the strength of the fifth force (additional to gravity). The constant
$\hat{m}_{\nu}$ is a free parameter of the model which determines
the size of the neutrino mass. (We take for simplicity all three neutrino
masses equal - or equivalently $m_{\nu}$ stands qualitatively for
the average over the neutrino species.) The special role of the neutrino
masses (as compared to quark and charged lepton masses) is motivated
at the particle physics level by the way in which neutrinos get masses
\cite{wetterich_growing_2007}. Growing neutrino quintessence with
a sufficiently large negative value of $\beta$ successfully relates
the present dark energy density and the mass of the neutrinos. The
evolution of the cosmon is effectively stopped once neutrinos become
non-relativistic. Dark energy becomes important now because neutrinos
become non-relativistic in a rather recent past, at typical redshifts
of about $z=5$ \cite{mota_neutrino_2008}. In this way, the ``why
now problem'' is resolved in terms of a ``cosmic trigger event''
induced by the change in the effective neutrino equation of state,
rather than by relying on the fine tuning of the scalar potential.
This differs from other mass varying neutrino cosmologies (usually
known as MaVaN's) \cite{brookfield_cosmology_2007,la_vacca_mass-varying_2013,bi_cosmological_2005,fardon_dark_2004,kaplan_neutrino_2004,spitzer_stability_2006,takahashi_speed_2006}.
Some of the observational consequences of those models were studied
in \cite{la_vacca_mass-varying_2013,kaplan_neutrino_2004} and more
recently a new scalar field - neutrino coupling that produces viable
cosmologies was proposed in \cite{simpson_dark_2016}. A viable cosmic
background evolution of growing neutrino quintessence offers interesting
prospects of a possible observation of the neutrino background.

The case in which the coupling $\beta$ is constant has been largely
investigated in literature at the linear level \cite{mota_neutrino_2008},
in semi-analytical non-linear methods \cite{wintergerst_clarifying_2010,wintergerst_very_2010,brouzakis_nonlinear_2011},
joining linear and non-linear information to test the effect of the
neutrino lumps on the cosmic microwave background \cite{pettorino_neutrino_2010}
and within N-Body simulations \cite{ayaita_neutrino_2013,ayaita_structure_2012,baldi_oscillating_2011,ayaita_nonlinear_2016}.
For the values of $\beta$ ($\beta\gtrsim10^{3}$) needed for dark
energy to dominate today, the cosmic neutrino background is clumping
very fast. Large and concentrated neutrino lumps form and induce very
substantial backreaction effects. These effects are so strong that
the deceleration of the evolution of the cosmon gets too weak, making
it difficult to obtain a realistic cosmology \cite{fuhrer_backreaction_2015}.

In this paper we instead consider the case in which the neutrino-cosmon
coupling $\beta(\phi)$ depends on the value of the cosmon field and
increases with time. In a particle physics context this has been motivated
\cite{wetterich_growing_2007} by a decrease with $\phi$ of the heavy
mass scale (B-L-violating scale) entering inversely the light neutrino
masses. In this scenario $\beta(\phi)$ has not been large in all
cosmological epochs - the present epoch corresponds to a crossover
where $\beta$ gets large. A numerical investigation \cite{baldi_oscillating_2011}
of this type of model has revealed compatibility with observations
for the case of a present neutrino mass $m_{\nu,0}=0.07$ eV. In the
present paper we investigate the dependence of cosmology on the value
of the neutrino mass by varying the parameter $\hat{m}_{\nu}$ in
eq. \ref{eq: mass_def}. For large neutrino masses we find a
qualitative behavior similar to the case of a constant neutrino-cosmon
coupling $\beta$, with difficulties to obtain a realistic cosmology.
In contrast, for small neutrino mass, the neutrino lumps form and
dissolve, with small influence on the overall cosmological evolution.
In this case, the neutrino-induced gravitational potentials are found
to be much smaller than the ones induced by dark matter. As we will
discuss in this paper, it will not be easy to find observational signals
for the neutrino lumps. In-between the regions of small and large
neutrino masses we expect a transition region for intermediate neutrino
masses where, by continuity, observable effects of the neutrino lumps
should show up.


\subsubsection{Growing neutrinos with varying coupling}

We consider here cosmologies in which neutrinos have a mass that varies
in time, along the framework of ``varying growing neutrino models''
\cite{wetterich_growing_2007}. As long as neutrinos are relativistic,
the coupling is inefficient and the dark energy scalar field $\phi$
rolls down a potential, as in an early dark energy scenario. As the
neutrino mass increases with time, neutrinos become non-relativistic,
typically at a relatively late redshift $z\approx4-6$ \cite{pettorino_neutrino_2010}.
This influences the evolution of $\phi$, which feels the effect of
neutrinos via a coupling to the neutrino mass $m_{\nu}(\phi)$. The
evolution of the scalar field slows down and practically stops, such
that the potential energy of the cosmon behaves almost as a cosmological
constant at recent times. In other words, in these models the cosmological
constant behavior observed today is related to a cosmological trigger
event (i.e. neutrinos becoming non-relativistic) and the present dark
energy density is directly connected to the value of the neutrino
mass. In the following we will detail the formalism and equations
used to describe the cosmological evolution of the model.

We start with the linearized Friedman-Lemaitre-Robertson-Walker metric
in the Newtonian gauge: 
\begin{equation}
ds^{2}=-(1+2\Psi)dt^{2}+a^{2}(1-2\Phi)d\mathbf{x}^{2}\,\,.
\end{equation}
Moreover, we use a quasi-static approximation for sub-horizon scales
($H/k\ll1$) which allows us to neglect time derivatives with respect
to spatial ones. Then the quasi-static, first-order perturbed Einstein
equations, are the Poisson equation \cite{ma_cosmological_1994}

\begin{equation}
k^{2}\Phi=4\pi Ga^{2}\delta T_{0}^{0}\,\,,\label{eq:Poisson}
\end{equation}
and the ``stress'' equation 
\begin{equation}
k^{2}(\Phi-\Psi)=12Ga^{2}(\bar{\rho}+\bar{P})\sigma\,\,,\label{eq:stress-phipsi}
\end{equation}
where $\delta T_{0}^{0}$ is the perturbation of the $0-0$ component
of the energy momentum tensor $T_{\mu\nu}$ and $\sigma$ is the anisotropic
stress of the fluid which depends on the traceless component of the
spatial part of the energy-momentum tensor, $T_{j}^{i}-\delta_{j}^{i}T_{k}^{k}/3$.
This stress tensor is in our case only important for relativistic
particles (i.e. the neutrinos). The source term of the Poisson equation
\ref{eq:Poisson} will contain contributions from all matter
species (dark matter \& neutrinos) and from the cosmon field. It is
proportional to the total density contrast $\delta\rho_{t}=\delta\rho_{\nu}+\delta\rho_{m}+\delta\rho_{\phi}$.

The cosmon field can be described through a Lagrangian in the standard
way 
\begin{equation}
-\mathcal{L_{\phi}}=\frac{1}{2}\partial^{\nu}\phi\partial_{\nu}\phi+V(\phi)
\end{equation}
where for this work we choose an exponential potential $V(\phi)\propto e^{-\alpha\phi}$.
The field dependent mass (eq.\ref{eq: mass_def}) allows for
an energy-momentum transfer between neutrinos and the cosmon, which
is proportional to the trace of the energy momentum tensor of neutrinos
$T_{(\nu)}$ and to a coupling parameter $\beta(\phi)$ 
\begin{align}
\nabla_{\eta}T_{(\phi)}^{\mu\eta}= & +\beta(\phi)T_{(\nu)}\partial^{\mu}\phi\,\,\,,\label{eq:continuity-phi}\\
\nabla_{\eta}T_{(\nu)}^{\mu\eta}= & -\beta(\phi)T_{(\nu)}\partial^{\mu}\phi\,\,\,.\label{eq:continuity-nu}
\end{align}


The cosmon is the mediator of a fifth force between neutrinos, acting
at cosmological scales. Its evolution is described by the Klein-Gordon
equation sourced by the trace of the energy-momentum tensor $T_{(\nu)}$
of the neutrinos,

\begin{equation}
\nabla_{\mu}\nabla^{\mu}\phi-V'(\phi)=\beta(\phi)T_{(\nu)}.\label{eq:klein-gordon-equation}
\end{equation}
As long as the neutrinos are relativistic ($T_{(\nu)}=0)$ the source
on the right hand side vanishes. During this time, the coupling has
no effect on the evolution of $\phi$. While the potential term $\sim V'$
drives $\phi$ towards larger values, the term $\sim\beta$ has the
opposite sign and stops the evolution effectively once $\beta T_{(\nu)}$
equals $V'$. The trace of the energy momentum tensor $T_{\nu}$,
entering eq.\ref{eq:klein-gordon-equation} is equal to: 
\begin{equation}
T_{\nu}=m_{\nu}(\phi)\tilde{n}(\phi)\label{eq:trace}
\end{equation}
where $\tilde{n}_{\nu}(\phi)=n_{\nu}(\phi)/\gamma$ is the ratio of
the number density of neutrinos $n_{\nu}$, divided by the relativistic
$\gamma$ factor. Eq.\ref{eq:trace} is valid for both relativistic
and non-relativistic neutrinos. Here we consider a coupling $\beta$
between neutrino particles and the quintessence scalar field $\phi$
as a field dependent quantity: 
\begin{equation}
\beta(\phi)\equiv-\frac{1}{\phi_{c}-\phi}\,.\label{eq:beta-of-phi}
\end{equation}
From eq.\ref{eq: mass_def} the neutrino mass is then given
by: 
\begin{equation}
m_{\nu}(\phi)=\frac{\bar{m}_{\nu}}{\phi_{c}-\phi}\,.\label{eq:mnu-of-phi}
\end{equation}
Here $\phi_{c}$ denotes the asymptotic value of $\phi$ for which
$\beta$ and $m_{\nu}(\phi)$ would formally become infinite. By an
additive shift in $\phi$ it can be set to an arbitrary value, e.g.
$\phi_{c}=0$. We consider the range $\phi<\phi_{c}$. The divergence
of $\beta$ for $\phi\rightarrow\phi_{c}$ in eq.\ref{eq:beta-of-phi}
is not crucial for the results of the present paper - $\beta$ and
$m_{\nu}$ never increase to large values, such that the immediate
vicinity of $\phi_{c}$ plays no role.

The coupling induces a total force acting on neutrinos given by $\nabla(\Phi_{\nu}+\beta\delta\phi)$
and appearing in the corresponding Euler equation \cite{pettorino_neutrino_2010},
as usual in coupled cosmologies \cite{baldi_hydrodynamical_2010}.
For values $2\beta^{2}>1$ the fifth force induced on neutrinos by
the cosmon becomes larger than the gravitational attraction. For the
large values of $|\beta|\approx10^{2}$ reached during the cosmological
evolution, the attraction induced by the cosmon gives rise to the
formation of neutrino lumps. As shown in \cite{mota_neutrino_2008,pettorino_neutrino_2010}
this represents the major difficulty encountered within growing neutrino
models and also, simultaneously, one of its clearest predictions with
respect to alternative dark energy models: the presence of neutrino
lumps at scales of $\approx10$ Mpc or even larger, depending on the
details of the model \cite{mota_neutrino_2008}. Since the attractive
force between neutrinos is $10^{4}$ times bigger than gravity, therefore
also the dynamical time scale of the clumping of neutrino inhomogeneities
is a factor $10^{4}$ faster than the gravitational time scale. Even
the tiny inhomogeneities in the cosmic neutrino background grow very
rapidly non-linear. The impact of such structures, has been shown
to depend crucially on the strength of backreaction effects \cite{ayaita_structure_2012,ayaita_nonlinear_2016}.
For constant coupling, the effect of backreaction is strong and can
lead to neutrino lumps with rapidly growing concentration, reaching
values of the gravitational potential which exceed observational constraints.
The effect is so strong that it is able to destroy the oscillatory
effect first encountered in \cite{baldi_hydrodynamical_2010}, in
which neutrino lumps were forming and then dissipating. No realistic
cosmology has been found in this case \cite{fuhrer_backreaction_2015}.
With the varying coupling of eq.\ref{eq:beta-of-phi} a similar
behavior will be found for large neutrino masses. For small neutrino
masses the oscillatory effects will be dominant and realistic cosmologies
seem possible \cite{ayaita_nonlinear_2016}.




%\subsection{Non-universal couplings in EFT}
%\comm{delete this paragraph}
%Paper by Gleyzes
%\section{Non-local gravity}
%\begin{itemize}
%\item Non-local terms in the action, like inverses of the box operator acting
%on the Ricci scalar, are motivated by quantum gravity corrections
%\item Non-local expressions can be localized in terms of scalar fields.
%\item There are the same number of parameters as in LCDM, instead of $\Lambda$
%there is a mass parameter.
%\item Seems to give a very good fit to present observations.
%\end{itemize}

%
%\begin{itemize}
%\item Differentiate between linear perturbations in different eras, just shortly
%\item Linear perturbations in matter dominated eras, newtonian gauge
%\item Fluid equations, full and linearized
%\end{itemize}


%----------------------------------------------------------------------------------------

%\section{Early Universe}
%\comm{We can probably delete this whole section}
%\subsection{Cosmic Microwave Background Radiation}
%\begin{itemize}
%\item Short introduction and importance of CMB
%\item Important constraints on parameters coming from CMB
%\item Constrain CDM alone
%\item Constrain initial power amplitude and tilt
%\item Constrain relativistic degrees of freedom
%\end{itemize}
%
%\subsection{Inflation} \comm{I would delete this}
%\begin{itemize}
%\item Inflation as a paradigm
%\item Flatness and horizon problems
%\item Inflation produces almost scale invariant spectrum
%\end{itemize}





