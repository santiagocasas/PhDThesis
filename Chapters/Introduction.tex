% Chapter 1

\chapter{Standard Cosmology \label{Overview}} % Main chapter title

 % For referencing the chapter elsewhere, use \ref{Chapter1} 

%----------------------------------------------------------------------------------------

% Define some commands to keep the formatting separated from the content 
\newcommand{\keyword}[1]{\textbf{#1}}
\newcommand{\tabhead}[1]{\textbf{#1}}
\newcommand{\code}[1]{\texttt{#1}}
\newcommand{\file}[1]{\texttt{\bfseries#1}}
\newcommand{\option}[1]{\texttt{\itshape#1}}

%----------------------------------------------------------------------------------------
%

Einstein's General Relativity is the theoretical framework used in cosmology to explain the
composition and the evolution of the Universe.
The most impressive fact about this 100-year-old theory is that
when it was formulated, there was no real experimental need for it, besides maybe the
perihelion of Mercury, and it was more than a decade later, thanks to observations
by Hubble and Slipher, that physicists and astronomers where convinced that there were objects much farther away from
our galaxy and that these objects were receding away from us, with velocities proportional
to their distances. At this point the field of physical cosmology was born and the theoretical basis for it,
based on General Relativity, was developed by Lema\^{\i}tre, Friedmann, and Einstein himself among other
notable scientists.

One century later, General Relativity has passed numerous very stringent tests; from laboratory experiments (\cite{cite}) , to
low orbit tests (\cite{cite}) and solar system tests (\cite{cite}), to pulsar timing tests and the recent
exciting first detection of gravitational waves (\cite{cite}).
It is impressive that a theory that was formulated on the grounds of some very basic
principles, has proven to be so accurate across several orders of magnitude in scales.

To explain the accelerated expansion of the Universe, which was observationally verified almost 20 years ago,
(\cite{cite, supernova, 1998}), Einstein's General Relativity needs to invoke a cosmological constant, which despite the fact that
it is allowed by the theory (see Lovelock's theorem below in\cref{sub:Einstein-Hilbert} ),
possesses many unsatisfactory properties from the classical and quantum point of view. 
We will review this issue in \cref{sub:CC-problem} below.

In order to provide some context for the main results of this dissertation, we will
provide in the following sections a very brief overview of General Relativity and 
the standard cosmological model. 
In \cref{sec:GR-framework} we will review the main principles and the mathematical 
formulation of General Relativity, its field equations and
its linearized Newtonian limit.
In \cref{sec:Standard-LCDM} we will deal with the composition of the Universe,
its evolution and the standard cosmological scenario.


\section{The framework of General Relativity \label{sec:GR-framework}}

\subsection{The equivalence principle and geometry}

For Einstein, Special Relativity (SR) was unsatisfactory and didn't represent a complete theory, because it dealt with 
intertial frames and under the influence of
a gravitational field, objects would be accelerated. Moreover due to the equivalence of mass and energy in SR, 
an object with a high kinetic energy in horizontal direction would have to be accelerated differently towards the Earth, than an object
with a smaller velocity, therefore violating Newtonian observations, which claim that all test bodies experience
the same acceleration in a gravitational field, regardless of its velocity or composition.
This observation that the inertial and gravitational masses must be equivalent was of striking significance for Einstein, 
which elevated it to a guiding principle in the construction of a relativistic theory of gravity (see \cite{pais, wald, bartelmann}).

The really ingenious step was connecting this principle with differential geometry. Then, instead of the rigid 
Euclidean picture, space and time would become a dynamical 4-dimensional structure, described by differential manifolds.
Gravity would curve this space-time, but in order to maintain the equivalence principle, it is always possible to transform away gravity 
and map it to a flat Euclidean space, which is precisely one of the properties of a differentiable manifold.
If one studies the consequences of this simple principle, it leads, without the need of formulating a theory, 
to concepts like gravitational redshift and gravitational light deflection. 
Constructing from there a fully-fledged non-linear theory of gravity took Einstein more than 10 years, with the (maybe indirect) help
of very bright mathematicians like Grossmann, Levi-Civita and Hilbert.

In more modern terms, we can say that General Relativity is a theory of 
a dynamical tensor field, the metric $g_{\mu \nu}$, which defines the lengths 
of space-time intervals $ds^2 = g_{\mu \nu} dx^{\mu} dx^{\nu}$ and which is invariant under diffeomorphisms. 
All particles and fields couple to the metric $g$ in a universal way.
This means that the equations of motion and all the physical properties do not depend on the chosen coordinates.
Diffeomorphism invariance is a very important symmetry that has to be respected if one wishes to construct
extensions of gravity. 
However, at the smallest (Planck length) and largest (super-horizon) scales, there are suggestions
that this symmetry might be broken in order to be able to construct a consistent theory of Quantum Gravity
(\cite{cite Rovelli}).



\subsection{The Einstein-Hilbert action and the field equations \label{sub:Einstein-Hilbert}}

Although Einstein postulated the field equations of General Relativity in a heuristic
form, they can be obtained by varying the so-called Einstein-Hilbert action
\begin{equation}\label{eq:Einstein-Hilbert action}
S = \frac{1}{16 \pi G} \int \textrm{d}x^4 \sqrt{-g}\left( R - 2\Lambda + \mathcal{L}_m \right) \quad 
\end{equation}
with respect to the metric $g_{\mu \nu}$.
Here, $R=R^\mu_\nu$ is the Ricci scalar and $R^\munu$ is the Riemann tensor (see standard GR textbooks like \cite{wald} for its definition), 
which are functions of derivatives of the metric. The volume element is defined as $\textrm{d}x^4 \sqrt{-g}$, where
$\sqrt{-g}$ is the square root of the determinant of the metric.
Furthermore, the cosmological constant is $\Lambda$, the gravitational constant is $G$ and $\mathcal{L}_m $ is the Lagrangian of matter and radiation species.
Then the variation ($\delta S / \delta g_{\mu \nu} = 0$), yields the field equations:
\begin{equation}\label{eq:Einstein-field-equations}
G_\munu + g_{\mu \nu} \Lambda = 8 \pi G T_{\mu \nu} \quad ,
\end{equation}
where the Einstein tensor is defined as $G_\munu \equiv R_{\mu \nu} - \frac{1}{2}g_{\mu \nu} R \;$ and the 
energy-momentum tensor is defined as:
\beeqp$
T_\munu = -\frac{2}{\sqrt{-g}} \frac{\delta \mathcal{L}_m}{\delta g_\munu}
$
What \cref{eq:Einstein-field-equations} expresses is that geometry (and therefore the dynamics of the metric) is sourced by 
the energy and momentum of the fields living on this manifold.
As it was famously expressed by \cite{Misner, Wheeler, Gravitation}:
"Matter tells space-time how to curve and space-time tells geometry how to move" .

Another important property of these field equations, is that due to a purely geometric property called the Bianchi identity, which
states that the covariant divergence of the Einstein tensor is identically zero:
$\nabla_{\mu} G^\munu = 0$, 
we can ensure that the energy-momentum tensor is locally conserved:
\beeqc$
\nabla_{\mu} T^\munu = 0
$
therefore one recovers all the well-known properties of classical and fluid mechanics.

Einstein's field equations are very complicated to solve analytically in its full glory. They consist on
10 coupled non-linear partial differential equations.
There are of course many interesting properties, solutions and implications of these equations, 
which are out of the scope of this work.
We refer the interested reader to very good textbooks like \cite{Carroll, Schultz, Misner, Thorne, Wheeler, Wald}.
The richness and complexity of this theory is one of the reasons it has to be so successful explaining 
star formation, gravitational collapse, black holes, gravitational waves and the evolution of the Universe.

\subsubsection{Lovelock's theorem and the uniqueness of General Relativity}

One might then ask if these field equations are unique, especially if as in our case, 
we are interested in testing these equations at the very largest scales of the Universe and we might 
be interested in modifying General Relativity in order to match current observations.
Thanks to a theorem by Vermeil and Cartan (\cite{cite Vermeil, Cartan, 1921}) 
and further simplified by Lovelock (\cite{1970, Lovelock}) we can state that in 4 dimensions,
the only divergence-free, rank-2 tensor $\mathcal{G}$, which depends on at most second derivatives
of the metric must be of the form:
\beeqp$
\mathcal{G} =  \alpha R_{\mu \nu} + \left( \Lambda - \frac{\alpha}{2} R \right) g_{\mu \nu}
$
Therefore, to ensure the correct Newtonian limit of the theory (Newtonian Poisson equation), we must set $\alpha \equiv 1$
and the proportionality constant between the Einstein tensor $G_\munu$  and the energy-momentum tensor
$T_\munu$ has to be set to $8 \pi G$.





\subsection{\revtext{Gravitational potentials}}

\begin{itemize}
\item Gravitational potentials $\Phi$ and $\Psi$.
\item Very simple explanation of Gauge choice \comm{probably don't need it}
\item Newtonian Gauge
\item Smallness of gravitational potentials \comm{what is this?}
\item Remark for later that in GR $\Psi=\Phi$
\item Weyl potential (lensing potential, null geodesics)
\item{The Newtonian limit} \comm{I put this here, because all items above are generally defined, not just in the Newtonian limit}
\end{itemize}

%----------------------------------------------------------------------------------------

\section{The standard cosmological model \label{sec:Standard-LCDM}}

\begin{itemize}
\item small discussion on how to define a cosmology
\item Observer along timelike geodesic and foliation of spacetime
\end{itemize}

\subsection{Friedmann equations}
\comm{We can probably make this section very short: 1 page}
 \begin{itemize}
 \item Friedmann-Lemaître-Robertson-Walker metric
 \item 00 and ii components of Einstein equation
 \item First and second Friedmann equations
 \item Several ways of writing Hubble function for different species
 \item Background expansion, , RDE, MDE, w(z)
 \end{itemize}

\subsection{Distances}
\comm{We can probably delete this whole section}
\begin{itemize}
\item definition of distances
\item definition of Hubble time 
\item Horizons?
\end{itemize}


\subsection{The $\Lambda$CDM \revtext{model}}

\begin{itemize}
\item Reasons for CDM
\item Reasons for Lambda
\item Baryons, neutrinos and other species
\end{itemize}

\subsection{The cosmological constant problem \label{sub:CC-problem}}

\begin{itemize}
\item small discussion on old and new cosmological constant problem
\item fine tuning, naturalness, basic definitions
\end{itemize}

\subsection{Linear perturbations}

\begin{itemize}
\item Differentiate between linear perturbations in different eras, just shortly
\item Linear perturbations in matter dominated eras, newtonian gauge
\item Fluid equations, full and linearized
\end{itemize}


%----------------------------------------------------------------------------------------

%\section{Early Universe}
%\comm{We can probably delete this whole section}
%\subsection{Cosmic Microwave Background Radiation}
%\begin{itemize}
%\item Short introduction and importance of CMB
%\item Important constraints on parameters coming from CMB
%\item Constrain CDM alone
%\item Constrain initial power amplitude and tilt
%\item Constrain relativistic degrees of freedom
%\end{itemize}
%
%\subsection{Inflation} \comm{I would delete this}
%\begin{itemize}
%\item Inflation as a paradigm
%\item Flatness and horizon problems
%\item Inflation produces almost scale invariant spectrum
%\end{itemize}





