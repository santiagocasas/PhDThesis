% Chapter 1

\chapter{Standard Cosmology \label{Overview}} % Main chapter title

 % For referencing the chapter elsewhere, use \ref{Chapter1} 

%----------------------------------------------------------------------------------------

% Define some commands to keep the formatting separated from the content 
\newcommand{\keyword}[1]{\textbf{#1}}
\newcommand{\tabhead}[1]{\textbf{#1}}
\newcommand{\code}[1]{\texttt{#1}}
\newcommand{\file}[1]{\texttt{\bfseries#1}}
\newcommand{\option}[1]{\texttt{\itshape#1}}

%----------------------------------------------------------------------------------------
%

Einstein's General Relativity is the theoretical framework used in cosmology to explain the
composition and the evolution of the Universe.
The most impressive fact about this 100-year-old theory is that
when it was formulated, there was no real experimental need for it, besides maybe the
perihelion of Mercury, and it was more than a decade later, thanks to observations
by Hubble and Slipher, that physicists and astronomers where convinced that there were objects much farther away from
our galaxy and that these objects were receding away from us, with velocities proportional
to their distances. At this point the field of physical cosmology was born and the theoretical basis for it,
based on General Relativity, was developed by Lema\^{\i}tre, Friedmann, and Einstein himself among other
notable scientists.

One century later, General Relativity has passed numerous very stringent tests; from laboratory experiments (\cite{cite}) , to
low orbit tests (\cite{cite}) and solar system tests (\cite{cite}), to pulsar timing tests and the recent
exciting first detection of gravitational waves (\cite{cite}).
It is impressive that a theory that was formulated on the grounds of some very basic
principles, has proven to be so accurate across several orders of magnitude in scales.

To explain the accelerated expansion of the Universe, which was observationally verified almost 20 years ago,
(\cite{cite, supernova, 1998}), Einstein's General Relativity needs to invoke a cosmological constant, which despite the fact that
it is allowed by the theory (see Lovelock's theorem below in\cref{sub:Einstein-Hilbert} ),
possesses many unsatisfactory properties from the classical and quantum point of view. 
We will review this issue in \cref{sub:CC-problem} below.

In order to provide some context for the main results of this dissertation, we will
provide in the following sections a very brief overview of General Relativity and 
the standard cosmological model. 
In \cref{sec:GR-framework} we will review the main principles and the mathematical 
formulation of General Relativity, its field equations and
its linearized Newtonian limit.
In \cref{sec:Standard-LCDM} we will deal with the composition of the Universe,
its evolution and the standard cosmological scenario.


\section{The framework of General Relativity \label{sec:GR-framework}}

\subsection{The equivalence principle and geometry}

For Einstein, Special Relativity (SR) was unsatisfactory and didn't represent a complete theory, because it dealt with 
intertial frames and under the influence of
a gravitational field, objects would be accelerated. Moreover due to the equivalence of mass and energy in SR, 
an object with a high kinetic energy in horizontal direction would have to be accelerated differently towards the Earth, than an object
with a smaller velocity, therefore violating Newtonian observations, which claim that all test bodies experience
the same acceleration in a gravitational field, regardless of its velocity or composition.
This observation that the inertial and gravitational masses must be equivalent was of striking significance for Einstein, 
which elevated it to a guiding principle in the construction of a relativistic theory of gravity (see \cite{pais, wald, bartelmann}).

The really ingenious step was connecting this principle with differential geometry. Then, instead of the rigid 
Euclidean picture, space and time would become a dynamical 4-dimensional structure, described by differential manifolds.
Gravity would curve this space-time, but in order to maintain the equivalence principle, it is always possible to transform away gravity 
and map it to a flat Euclidean space, which is precisely one of the properties of a differentiable manifold.
If one studies the consequences of this simple principle, it leads, without the need of formulating a theory, 
to concepts like gravitational redshift and gravitational light deflection. 
Constructing from there a fully-fledged non-linear theory of gravity took Einstein more than 10 years, with the (maybe indirect) help
of very bright mathematicians like Grossmann, Levi-Civita and Hilbert.

In more modern terms, we can say that General Relativity is a theory of 
a dynamical tensor field, the metric $g_{\mu \nu}$, which defines the lengths 
of space-time intervals $ds^2 = g_{\mu \nu} dx^{\mu} dx^{\nu}$ and which is invariant under diffeomorphisms. 
All particles and fields couple to the metric $g$ in a universal way.
This means that the equations of motion and all the physical properties do not depend on the chosen coordinates.
Diffeomorphism invariance is a very important symmetry that has to be respected if one wishes to construct
extensions of gravity. 
However, at the smallest (Planck length) and largest (super-horizon) scales, there are suggestions
that this symmetry might be broken in order to be able to construct a consistent theory of Quantum Gravity
(\cite{cite Rovelli}).



\subsection{The Einstein-Hilbert action and the field equations \label{sub:Einstein-Hilbert}}

Although Einstein postulated the field equations of General Relativity in a heuristic
form, they can be obtained by varying the so-called Einstein-Hilbert action
\begin{equation}\label{eq:Einstein-Hilbert action}
S = \frac{1}{16 \pi G} \int \textrm{d}x^4 \sqrt{-g}\left( R - 2\Lambda + \mathcal{L}_m \right) \quad 
\end{equation}
with respect to the metric $g_{\mu \nu}$.
Here, $R=R^\mu_\nu$ is the Ricci scalar and $R^\munu$ is the Riemann tensor (see standard GR textbooks like \cite{wald} for its definition), 
which are functions of derivatives of the metric. The volume element is defined as $\textrm{d}x^4 \sqrt{-g}$, where
$\sqrt{-g}$ is the square root of the determinant of the metric.
Furthermore, the cosmological constant is $\Lambda$, the gravitational constant is $G$ and $\mathcal{L}_m $ is the Lagrangian of matter and radiation species.
Then the variation ($\delta S / \delta g_{\mu \nu} = 0$), yields the field equations:
\begin{equation}\label{eq:Einstein-field-equations}
G_\munu + g_{\mu \nu} \Lambda = 8 \pi G T_{\mu \nu} \quad ,
\end{equation}
where the Einstein tensor is defined as $G_\munu \equiv R_{\mu \nu} - \frac{1}{2}g_{\mu \nu} R \;$ and the 
energy-momentum tensor is defined as:
\beeqp$
T_\munu = -\frac{2}{\sqrt{-g}} \frac{\delta \mathcal{L}_m}{\delta g_\munu}
$
What \cref{eq:Einstein-field-equations} expresses is that geometry (and therefore the dynamics of the metric) is sourced by 
the energy and momentum of the fields living on this manifold.
As it was famously expressed by \cite{Misner, Wheeler, Gravitation}:
"Matter tells space-time how to curve and space-time tells geometry how to move" .

Another important property of these field equations, is that due to a purely geometric property called the Bianchi identity, which
states that the covariant divergence of the Einstein tensor is identically zero:
$\nabla_{\mu} G^\munu = 0$, 
we can ensure that the energy-momentum tensor is locally conserved:
\beeqc$
\nabla_{\mu} T^\munu = 0
$
therefore one recovers all the well-known properties of classical and fluid mechanics.

Einstein's field equations are very complicated to solve analytically in its full glory. They consist on
10 coupled non-linear partial differential equations.
There are of course many interesting properties, solutions and implications of these equations, 
which are out of the scope of this work.
We refer the interested reader to very good textbooks like \cite{Carroll, Schultz, Misner, Thorne, Wheeler, Wald}.
The richness and complexity of this theory is one of the reasons it has to be so successful explaining 
star formation, gravitational collapse, black holes, gravitational waves and the evolution of the Universe.

\subsubsection{Lovelock's theorem and the uniqueness of General Relativity}

One might then ask if these field equations are unique, especially if as in our case, 
we are interested in testing these equations at the very largest scales of the Universe and we might 
be interested in modifying General Relativity in order to match current observations.
Thanks to a theorem by Vermeil and Cartan (\cite{cite Vermeil, Cartan, 1921}) 
and further simplified by Lovelock (\cite{1970, Lovelock}) we can state that in 4 dimensions,
the only divergence-free, rank-2 tensor $\mathcal{G}$, which depends on at most second derivatives
of the metric must be of the form:
\beeqp$
\mathcal{G} =  \alpha R_{\mu \nu} + \left( \Lambda - \frac{\alpha}{2} R \right) g_{\mu \nu}
$
Therefore, to ensure the correct Newtonian limit of the theory (Newtonian Poisson equation), we must set $\alpha \equiv 1$
and the proportionality constant between the Einstein tensor $G_\munu$  and the energy-momentum tensor
$T_\munu$ has to be set to $8 \pi G$.

This theorem has to fundamental implications for Cosmology.
First of all, it states that the Cosmological Constant (CC) has to appear in the classical theory of gravity,
therefore, if ---as we will see below in \cref{sub:CC-problem})--- we are not satisfied with the CC as an explanation
of the accelerated expansion of the Universe, we still have to explain why $\Lambda$ disappears from the Einstein's field equations.
Second, this theorem gives a sort of roadmap to look for modifications of gravity, which can account for the late-time expansion of the Universe.
One option is to go beyond 4 dimensions (as in DGP models \cite{DGP models}),
violate the conservation of the energy-momentum tensor with some exotic species or
add extra degrees of freedom, since with the metric alone we can't construct a more general theory.
This last option has led to the recent interest in modified gravity theories with a scalar field (see \cite{cite some scalar field}), 
with vector fields (see \cite{cite}) and with the addition of several metric (tensor) fields (see \cite{cite}). 


%----------------------------------------------------------------------------------------

\section{The standard cosmological model \label{sec:Standard-LCDM}}

The cosmological principle states
that no observer in the Universe is special and that each observer sees the Universe in the same way independently of
spatial rotations, therefore space-time has to be described by a homogeneous and isotropic metric $g_\munu$.
If furthermore, we can define a foliation of space-time, in which there is a preferred timelike direction, orthogonal
to the spatial hypersurfaces, we end up with a metric which solves the Einstein's field equations \ref{eq:Einstein-field-equations};
the so-called Friedmann-Lema\^{\i}tre-Robertson-Walker (FLRW) metric:
\beeqc$\label{eq:FLRW-full}
ds^2 = g_\munu dx^\mu dx^\nu = -dt^2 + a^2 (t) d\varsigma^2 
$
where $a$ is the scale factor, $t$ the cosmic time coordinate, and $d\varsigma^2$ is the time-independent spatial metric:
\beeqc$\label{eq:FLRW-spatial}
d\varsigma^2  = \gamma_{i j} dx^i dx^j = \frac{dr^2}{1-kr^2} + r^2(d\theta^2 + \sin^2 \theta d \phi^2)
$ 
where $r$ is the radial coordinate, $\theta$ the polar angle and $\phi$ the azimuthal angle.
The curvature $k$ which can be 0, 1 or -1, corresponds to Universes which are flat, closed or open, respectively.
Due to the stringent constraints on $k$ given by recent observations,
we will use for the rest of this work only a flat geometry with $k=0$.
Also it is standard convention that "Greek" indices run from 0 to 3, as in \cref{eq:FLRW-full},
while "Latin" indices run from 1 to 3 as in equations involving only spatial coordinates, like \cref{eq:FLRW-spatial}.




\subsection{The Friedmann equations \label{sub:Friedmann-eqs}}

The FLRW metric \cref{eq:FLRW-full} due to its scale factor, which is dependent on time, implies immediately that the Universe can expand
in its spatial coordinates. The equations describing the evolution
of the scale factor are called the \emph{Friedmann} equations. To derive them,
we need to introduce in the right hand side of Einstein's equations \cref{eq:Einstein-field-equations} an energy-momentum tensor of a perfect fluid:
\beeqc$
T^\munu = (\rho + p)u^\mu u^\nu + p g^\munu
$
which is justified since at cosmological scales, we expect the background matter in the Universe to be absent of dissipative and viscous forces.
The density $\rho$ and the pressure $p$ are the sum of the densities and pressure terms of all matter and radiation species in the universe.
If we write down now the (00) and ($i i$) components of the Einstein's field equations \cref{eq:Einstein-field-equations}, computing
the Riemann and Ricci tensors of a FLRW metric, 
we end up with two ordinary differential equations for the scale factor $a(t)$:
\beeqal$
\frac{\dot a}{a} &= \frac{8 \pi G}{3} \rho  \label{eq:Friedmann-1st}\\
\frac{\ddot a }{a} &= -\frac{4 \pi G}{3} (\rho +3 p) \label{eq:Friedmann-2nd}
$
These are the so-called Friedmann equations, which define the evolution of the Hubble function defined as:
\beeqp$
H(t) \equiv \frac{\dot a(t)}{a(t)}
$

The critical density of the Universe is defined as:
\beeqc$
\rho_{cr} = \frac{3 H^2}{8 \pi G}
$
which is the critical density that an Universe with zero curvature $k=0$ would have according to the observed value of the Hubble function.
So, for each species in the Universe, with energy density $\rho_i$, we can define the energy density fraction as:
\beeqc$
\Omega_i (t) = \frac{\rho_i (t)}{\rho_cr (t)}
$ 
so that the first Friedmann equation \cref{eq:Friedmann-1st} (for a flat Universe) can be written as:
\beeqp$
\sum_i \Omega_i (t) = 1
$

Another important quantity for each matter species is its equation of state:
\beeqc$
w \equiv \frac{p}{\rho}
$
which for species in which the pressure is not-negligible modifies considerably their dominance in the energy budget as a function of time.


\subsection{The $\Lambda$CDM model \label{sub:LCDM}}

As we have seen before, the field of Cosmology is relatively new, with less than 100 years of theoretical development
and even much less time of real observational progress.
Up to 1965?, with the discovery of the Cosmic Microwave Background (CMB) radiation by \cite{Penzias and Wilson},
cosmology was a mere speculative science. Only with the confirmation of the "Hot Big Bang paradigm" by the measurement of the radiation left over
380,000 years after the Big Bang, is that physicists started to make precise descriptions about the composition and 
the evolution of the matter species composing the Universe. The first CMB observations clarified that the Universe started
as a hot plasma of electrons, protons and photons, whose interactions in an expanding space-time matched very well the observed properties
of the present Universe.

However, there was always a piece missing in the puzzle, since the observed densities of visible matter in the Universe were too low.
For many years, there were big discussions about open and closed Universes and about invisible forms of matter.
Only with the amazing development of galaxy surveys in the 1980's and precise measurements of stars velocities in the Milky Way and other galaxies
and the development of cosmological N-body simulations in the 1990's, most cosmologists were convinced that there had to be a form of matter in the Universe
that does not interact with baryons or with electromagnetic fields, but it interacts in a standard way with gravitational fields (\cite{cite;
recent peebles, Dolag DM review, reviews, books}).
They called it "Cold Dark Matter" (CDM) because of its apparent vanishing pressure and its preference for clustering into big structures.

However, there were still some inconsistencies between the growth of structures in a pure 'CDM+baryons' model, predicted
by simulations and the observed structures in the Universe (see \cite{reviews Dolag, peacock}), so that many scientists were claiming for the addition
of a cosmological constant. However, this claims were not significant until in 1998 the teams led by \cite{Perlmutter and ..} used Supernova Type Ia
observations to prove that the Universe was experiencing a phase of accelerated expansion. This led to the re-introduction of the Cosmological Constant
and due to the theoretical problems posed by its so small observed value, to the development of a whole new 
field of research: Modified Gravity and Dark Energy.

This very rough and quick overview explains the name of the current standard cosmological model: the $\lcdm$ model. 
In this model, and according to the latest observations (see \cite{planck_collaboration_planck_2016-1})
 almost 70\% of the energy density of the Universe is composed by the Cosmological Constant
$\Lambda$, 25\% by Cold Dark Matter and less than 5\% by baryons. The remaining components are photons ---which are basically negligible today, despite the amount of light and radiation in the Universe--- and massive neutrinos,
whose mass and therefore its contribution to the "cosmic pie" has not been measured precisely enough yet.

Now we leave open the possibility that the accelerated expansion of the Universe is caused by a "Dark Energy" component, with 
an unknown equation of state $w_{DE}(z)$, where $w_{DE}(z) = -1$ would correspond to the Cosmological Constant. 
Then, together with the "standard" matter species and knowing the equation of state for each of them ($w=0$ for CDM and baryons and $w=1/3$ for radiation), we can write down
the evolution of the Hubble parameter as a function of redshift $z \equiv 1/a - 1 $ :
\beeqalsp$
H^2 (z) &= H_0 \left(  \Omega_c (1+z)^3 + \Omega_b (1+z)^3  + \Omega_r (1+z)^4   
\phantom{\exp \left[ \int_{0}^{z}  \frac{3}{1} \right]} \right. \\
      & \left.  +\Omega_{DE}  \exp \left[ \int_{0}^{z} \dx{}{\tilde z} \frac{3(1+w_{DE}(\tilde z))}{1+\tilde z} \right]     \right)
$

This is the equation that defines the background evolution of the Universe. From the Hubble function, one can obtain measurable 
cosmological distances,
the age of the Universe and the size of the observable Universe, also called the horizon.


\subsection{Distances}

Since the Universe is expanding and we measure astrophysical objects using mainly electromagnetic radiation (except after 2016 \cite{cite LIGO}, 
where Gravitational Wave Astronomy was born), it is important to define certain distances that can be measured and which are not always intuitive.
Light follows null-geodesics, so that under an FLRW metric as in \cref{eq:FLRW-full}, 
light satisfies the equation:
\beeqc$
- c dt^2 + a^2 (t) d\varsigma^2 = 0 
$
where we have recovered the speed of light $c$ to avoid confusion.
Solving for $\varsigma$ and therefore integrating this equation, defines the comoving distance $d_c$:
\beeqp$
d_c \equiv \int_0^{\varsigma_1} \mathrm{d}\varsigma = - \int_{t_0}^{t^1} \frac{c}{a(t)} \dx{}{t} 
$
Since $H = (\mathrm{d}a/\mathrm{d}t)/a = (\mathrm{d}(1/(1+z))/\mathrm{d}t)/(1/(1+z)) $, then $\mathrm{d} t = -\mathrm{d}z / ((1+z) H) $,
so that the comoving distance can be defined as:
\beeqp$
d_c (z) = \int_{0}^{z} \frac{\dx{}{\tilde z}}{H(\tilde z)}
$

\subsubsection{Angular Diameter Distance}

\subsection{The cosmological constant problem \label{sub:CC-problem}}

\begin{itemize}
\item small discussion on old and new cosmological constant problem
\item fine tuning, naturalness, basic definitions
\end{itemize}

\subsection{\revtext{Gravitational potentials}}

\begin{itemize}
	\item Gravitational potentials $\Phi$ and $\Psi$.
%	\item Very simple explanation of Gauge choice \comm{probably don't need it}
	\item Newtonian Gauge
%	\item Smallness of gravitational potentials \comm{what is this?}
	\item Remark for later that in GR $\Psi=\Phi$
	\item Weyl potential (lensing potential, null geodesics)
	\item{The Newtonian limit}
\end{itemize}

%
%\begin{itemize}
%\item Differentiate between linear perturbations in different eras, just shortly
%\item Linear perturbations in matter dominated eras, newtonian gauge
%\item Fluid equations, full and linearized
%\end{itemize}


%----------------------------------------------------------------------------------------

%\section{Early Universe}
%\comm{We can probably delete this whole section}
%\subsection{Cosmic Microwave Background Radiation}
%\begin{itemize}
%\item Short introduction and importance of CMB
%\item Important constraints on parameters coming from CMB
%\item Constrain CDM alone
%\item Constrain initial power amplitude and tilt
%\item Constrain relativistic degrees of freedom
%\end{itemize}
%
%\subsection{Inflation} \comm{I would delete this}
%\begin{itemize}
%\item Inflation as a paradigm
%\item Flatness and horizon problems
%\item Inflation produces almost scale invariant spectrum
%\end{itemize}





