\global\long\def\vk#1{\mathbf{#1}}


\global\long\def\pkn{(\mathbf{k},\eta)}


\global\long\def\pkdn{(\mathbf{k};\eta)}


\global\long\def\ppn#1{(\mathbf{#1},\eta)}


\global\long\def\ppt#1{(\mathbf{#1},\tau)}


\global\long\def\pkc{(\vk k,\mathcal{X})}


\global\long\def\pkdc{(\vk k;\mathcal{X})}


\global\long\def\vpa#1{\varphi_{#1}}


\global\long\def\psq#1#2#3{(\mathbf{#1}+\mathbf{#2}+\mathbf{#3})}


\global\long\def\pmq#1#2#3{(\mathbf{#1}-\mathbf{#2}-\mathbf{#3})}


\global\long\def\pcq#1#2#3{(\mathbf{#1},\mathbf{#2},\mathbf{#3})}

\global\long\def\paralone#1{\frac{\partial}{\partial{#1}}}

\global\long\def\paralonemix#1#2{\frac{\partial}{\partial{#1}\partial{#2}}}

\global\long\def\parder#1#2{\frac{\partial{#1}}{\partial{#2}}}

\global\long\def\pardersq#1#2{\frac{\partial^2{#1}}{\partial{#2}^2}}

\global\long\def\pardermix#1#2#3{\frac{\partial^2{#1}}{\partial{#2}\partial{#3}}}

\def\inserteq#1#2{\begin{equation}{#1}\label{#2}\end{equation}}   


\global\long\def\colv#1#2{\begin{pmatrix}#1\\
#2 
\end{pmatrix}}

\def\beeq$#1${\begin{equation}#1\end{equation}}

%\def\beeql#1$#2${\begin{equation} \label{#1} #2\end{equation}}

\def\beeqc$#1${\begin{equation} #1 \;\;\;, \end{equation}}

\def\beeqp$#1${\begin{equation} #1 \;\;\;. \end{equation}}

%       BEGIN EQUATION MODE
\newcommand{\beq}{\begin{equation}} 
%       BEGIN EQUATION MODE WITH LABEL
\newcommand{\beql}[1]{\begin{equation}\label{#1}}
%       END EQUATION MODE
\newcommand{\eeq}{\end{equation}}
%       END EQUATION MODE WITH A PERIOD
\newcommand{\eeqp}{\;\;\;.\end{equation}}
%       END EQUATION MODE WITH A COMMA
\newcommand{\eeqc}{\;\;\;,\end{equation}}


\global\long\def\curH{\mathcal{H}}


\global\long\def\pet{\partial_{\eta}}


\global\long\def\d{\mathrm{d}}


\global\long\def\mychi{\mathcal{X}}

\newcommand{\Li}{\mathcal{L}}


\global\long\def\comment#1{\textcolor{red}{#1}}

\usepackage{verbatim}
\newcommand{\HS}{\mathrm{HS}}
\newcommand{\HMG}{\mathrm{HMG}}
\newcommand{\HGR}{\mathrm{HGR}}
\newcommand{\LMG}{\mathrm{LMG}}
\newcommand{\GRNL}{\mathrm{GR-NL}}
\newcommand{\LIN}{\mathrm{L}}
\newcommand{\cnl}{c_{\mathrm{nl}}}
\newcommand{\Pobs}{P_{\mathrm{obs}}}
\newcommand{\lmax}{\ell_{\mathrm{max}}}
\newcommand{\lAs}{\ell \mathcal{A}_{s}}
\newcommand{\planck}{{\it Planck}}


\newcommand{\lcdm}{\Lambda\mathrm{CDM}}
%\newcommand{\curH}{\mathcal{H}}
\def\vdotov#1{\frac{\dot{#1}}{#1}}

\newcommand*{\srr}{\srvtxt}
\newcommand*{\scc}{\scmtxt}


% Alter some LaTeX defaults for better treatment of figures:
    % See p.105 of "TeX Unbound" for suggested values.
    % See pp. 199-200 of Lamport's "LaTeX" book for details.
    %   General parameters, for ALL pages:
    \renewcommand{\topfraction}{0.9}	% max fraction of floats at top
    \renewcommand{\bottomfraction}{0.8}	% max fraction of floats at bottom
    %   Parameters for TEXT pages (not float pages):
    \setcounter{topnumber}{2}
    \setcounter{bottomnumber}{2}
    \setcounter{totalnumber}{4}     % 2 may work better
    \setcounter{dbltopnumber}{2}    % for 2-column pages
    \renewcommand{\dbltopfraction}{0.9}	% fit big float above 2-col. text
    \renewcommand{\textfraction}{0.07}	% allow minimal text w. figs
    %   Parameters for FLOAT pages (not text pages):
    \renewcommand{\floatpagefraction}{0.8}	% require fuller float pages
	% N.B.: floatpagefraction MUST be less than topfraction !!
    \renewcommand{\dblfloatpagefraction}{0.7}	% require fuller float pages

	% remember to use [htp] or [htpb] for placement

\def\l@section{\@dottedtocline{1}{1em}{2em}}
\def\l@subsection{\@dottedtocline{2}{2em}{4em}}
\def\l@subsubsection{\@dottedtocline{3}{3em}{5em}}

\usepackage[titletoc]{appendix}

\usepackage{colortbl}

\newcommand{\todoinl}[1]{\todo[inline]{#1}}

\newcommand{\Tstrut}{\rule{0pt}{2.6ex}}       % "top" strut
\newcommand{\Bstrut}{\rule[-0.9ex]{0pt}{0pt}} % "bottom" strut
\newcommand{\TBstrut}{\Tstrut\Bstrut} % top&bottom struts
\newcommand{\Hstrut}{\rule{2.6ex}{0pt}} 


\def\mnras{MNRAS}
\newcommand{\noun}[1]{\textsc{#1}}

\newcount\colveccount
\newcommand*\colvec[1]{
        \global\colveccount#1
        \begin{pmatrix}
        \colvecnext
}
\def\colvecnext#1{
        #1
        \global\advance\colveccount-1
        \ifnum\colveccount>0
                \\
                \expandafter\colvecnext
        \else
                \end{pmatrix}
        \fi
}